% ============================================================================
% BIBLIOGRAFÍA
% ============================================================================

\begin{thebibliography}{00}

\bibitem{bloom1984}
B. S. Bloom, ``The 2 sigma problem: The search for methods of group instruction as effective as one-to-one tutoring,'' \emph{Educational Researcher}, vol. 13, no. 6, pp. 4--16, 1984.

\bibitem{vanlehn2011}
K. VanLehn, ``The relative effectiveness of human tutoring, intelligent tutoring systems, and other tutoring systems,'' \emph{Educational Psychologist}, vol. 46, no. 4, pp. 197--221, 2011.

\bibitem{kulik2016}
J. A. Kulik and J. D. Fletcher, ``Effectiveness of intelligent tutoring systems: A meta-analytic review,'' \emph{Review of Educational Research}, vol. 86, no. 1, pp. 42--78, 2016.

\bibitem{ma2014}
W. Ma, O. O. Adesope, J. C. Nesbit, and Q. Liu, ``Intelligent tutoring systems and learning outcomes: A meta-analysis,'' \emph{Journal of Educational Psychology}, vol. 106, no. 4, pp. 901--918, 2014.

\bibitem{lee2024impact}
D. Lee et al., ``The impact of generative AI on higher education learning and teaching: A study of educators' perspectives,'' \emph{Computers and Education: Artificial Intelligence}, vol. 6, p. 100221, 2024.

\bibitem{kestin2025}
G. Kestin et al., ``AI tutoring outperforms active learning,'' \emph{Nature Human Behaviour}, 2025.

\bibitem{tack2022}
A. Tack and C. Piech, ``The AI teacher test: Measuring the pedagogical ability of blender and GPT-3 in educational dialogues,'' in \emph{Proc. 15th Int. Conf. on Educational Data Mining}, 2022.

\bibitem{macina2023}
J. Macina et al., ``Opportunities and challenges in neural dialog tutoring,'' in \emph{Proc. 16th Int. Conf. on Educational Data Mining}, 2023.

\bibitem{borchers2025}
C. Borchers et al., ``Comparing LLM-based tutors to experienced human tutors,'' in \emph{Proc. Int. Conf. on Artificial Intelligence in Education (AIED)}, 2025.

\bibitem{puech2024}
R. Puech et al., ``Towards pedagogical steering of LLMs for tutoring: A case with productive failure,'' \emph{arXiv preprint arXiv:2410.03781}, 2024.

\bibitem{scarlatos2025}
A. Scarlatos et al., ``Exploring the limitations of LLMs as intelligent tutoring systems,'' \emph{arXiv preprint arXiv:2504.05570}, 2025.

\bibitem{nwana1990}
H. S. Nwana, ``Intelligent tutoring systems: An overview,'' \emph{Artificial Intelligence Review}, vol. 4, no. 4, pp. 251--277, 1990.

\bibitem{woolf2009}
B. P. Woolf, \emph{Building Intelligent Interactive Tutors}. Burlington, MA: Morgan Kaufmann, 2009.

\bibitem{anderson1995}
J. R. Anderson et al., ``Cognitive tutors: Lessons learned,'' \emph{The Journal of the Learning Sciences}, vol. 4, no. 2, pp. 167--207, 1995.

\bibitem{corbett1994}
A. T. Corbett and J. R. Anderson, ``Knowledge tracing,'' \emph{User Modeling and User-Adapted Interaction}, vol. 4, no. 4, pp. 253--278, 1994.

\bibitem{aleven2009}
V. Aleven et al., ``A new paradigm for intelligent tutoring systems: Example-tracing tutors,'' \emph{Int. Journal of Artificial Intelligence in Education}, vol. 19, no. 2, pp. 105--154, 2009.

\bibitem{murray1999}
T. Murray, ``Authoring intelligent tutoring systems: An analysis of the state of the art,'' \emph{Int. Journal of Artificial Intelligence in Education}, vol. 10, pp. 98--129, 1999.

\bibitem{sottilare2012}
R. A. Sottilare et al., ``Design recommendations for intelligent tutoring systems,'' U.S. Army Research Laboratory, 2012.

\bibitem{graesser2004}
A. C. Graesser et al., ``AutoTutor: An intelligent tutoring system with mixed-initiative dialogue,'' \emph{IEEE Transactions on Education}, vol. 48, no. 4, pp. 612--618, 2004.

\bibitem{pardos2024}
Z. A. Pardos and S. Bhandari, ``Learning gain differences between ChatGPT and human tutors-generated hints,'' \emph{PLOS ONE}, vol. 19, no. 3, e0300722, 2024.

\bibitem{merrill1983}
M. D. Merrill, ``Component display theory,'' in \emph{Instructional-Design Theories and Models}, C. M. Reigeluth, Ed. Hillsdale, NJ: Lawrence Erlbaum, 1983, pp. 279--333.

\bibitem{merrill1991}
M. D. Merrill, L. Li, and M. K. Jones, ``Instructional transaction theory: An introduction,'' \emph{Educational Technology}, vol. 31, no. 6, pp. 7--12, 1991.

\bibitem{gagne1985}
R. M. Gagne, \emph{The Conditions of Learning and Theory of Instruction}, 4th ed. New York: Holt, Rinehart and Winston, 1985.

\bibitem{anderson2001}
L. W. Anderson and D. R. Krathwohl (Eds.), \emph{A Taxonomy for Learning, Teaching, and Assessing: A Revision of Bloom's Taxonomy of Educational Objectives}. New York: Longman, 2001.

\bibitem{koper2005}
R. Koper and C. Tattersall (Eds.), \emph{Learning Design: A Handbook on Modelling and Delivering Networked Education and Training}. Berlin: Springer, 2005.

\bibitem{derntl2012}
M. Derntl et al., ``The conceptual structure of IMS Learning Design does not impede its use for authoring,'' \emph{IEEE Transactions on Learning Technologies}, vol. 5, no. 1, pp. 74--86, 2012.

\bibitem{griffiths2005}
D. Griffiths et al., ``Learning design tools,'' in \emph{Learning Design}, R. Koper and C. Tattersall, Eds. Berlin: Springer, 2005, pp. 109--135.

\bibitem{berggren2005}
A. Berggren et al., ``Practical and pedagogical issues for teacher adoption of IMS Learning Design,'' \emph{Journal of Interactive Media in Education}, vol. 2005, no. 1, 2005.

\bibitem{neumann2008}
S. Neumann and R. Oberhuemer, ``User evaluation of a graphical modeling tool for IMS Learning Design,'' in \emph{Proc. 8th Int. Conf. on Advanced Learning Technologies}, IEEE, 2008, pp. 287--291.

\bibitem{ainsworth2002}
S. E. Ainsworth et al., ``Using edit distance algorithms to compare alternative approaches to ITS authoring,'' in \emph{Proc. Int. Conf. on Intelligent Tutoring Systems}, Springer, 2002.

\bibitem{vandeursen2000}
A. van Deursen et al., ``Domain-specific languages: An annotated bibliography,'' \emph{ACM SIGPLAN Notices}, vol. 35, no. 6, pp. 26--36, 2000.

\bibitem{fowler2010}
M. Fowler, \emph{Domain-Specific Languages}. Boston: Addison-Wesley, 2010.

\bibitem{ibm2024pdl}
IBM Research, ``Prompt Declaration Language (PDL),'' 2024. [Online]. Available: https://ibm.github.io/prompt-declaration-language/

\bibitem{zhang2025poml}
Y. Zhang, N. Chen, J. Xu, and Y. Yang, ``Prompt Orchestration Markup Language,'' \emph{arXiv preprint arXiv:2508.13948}, 2025.

\bibitem{pernias2025adl}
P. Pernias et al., ``ADL-based educational assistant architecture: Transforming pedagogical design into automated tutoring systems,'' in \emph{Proc. ICERI 2025}, 2025.

\bibitem{pernias2025adl2}
P. A. Pernías Peco and M. P. Escobar Esteban, ``ADL 2.0: A core specification framework for LLM-based assistants,'' \emph{arXiv preprint}, 2025.

\bibitem{white2023}
J. White et al., ``A prompt pattern catalog to enhance prompt engineering with ChatGPT,'' \emph{arXiv preprint arXiv:2302.11382}, 2023.

\bibitem{reynolds2021}
L. Reynolds and K. McDonell, ``Prompt programming for large language models: Beyond the few-shot paradigm,'' in \emph{CHI 2021 Extended Abstracts}. ACM, 2021, pp. 1--7.

\bibitem{holmes2019}
W. Holmes, M. Bialik, and C. Fadel, \emph{Artificial Intelligence in Education: Promises and Implications for Teaching and Learning}. Boston: Center for Curriculum Redesign, 2019.

\end{thebibliography}
