% ============================================================================
% II. ESTADO DEL ARTE EN ITS
% ============================================================================

\section{Sistemas de Tutoría Inteligente: Estado del Arte}

Esta sección revisa dos décadas de investigación en ITS (2005-2025), centrándose en la arquitectura, evidencia de efectividad, herramientas de autoría, y la transición hacia LLM.

\subsection{Arquitectura Clásica de ITS}

La arquitectura canónica de un ITS, establecida por Nwana \cite{nwana1990} y refinada en trabajos posteriores \cite{woolf2009}, comprende cuatro componentes interrelacionados:

\begin{enumerate}
    \item \textbf{Modelo de Dominio}: Representa el conocimiento experto que el sistema enseña.
    
    \item \textbf{Modelo del Estudiante}: Mantiene una representación del estado cognitivo del aprendiz: qué sabe, qué misconceptions tiene, y cómo progresa.
    
    \item \textbf{Modelo Pedagógico}: Contiene las estrategias instruccionales y decide qué acción tomar dado el estado del estudiante y el dominio.
    
    \item \textbf{Interfaz de Usuario}: Gestiona la comunicación entre el sistema y el estudiante.
\end{enumerate}

El \textit{student model} merece atención especial porque es el componente que diferencia un ITS de instrucción computarizada convencional. Técnicas como \textit{model tracing} \cite{anderson1995} comparan las acciones del estudiante contra un modelo cognitivo, identificando errores en tiempo real. \textit{Knowledge tracing} \cite{corbett1994} estima probabilísticamente el dominio de habilidades específicas.

\subsection{Evidencia de Efectividad}

La efectividad de los ITS está bien documentada mediante múltiples meta-análisis:

\textbf{VanLehn (2011)} \cite{vanlehn2011} comparó 54 estudios y encontró: tutoría humana $d = 0.79$; ITS \textit{step-based} $d = 0.76$; ITS \textit{substep-based} $d = 0.40$; CAI \textit{answer-based} $d = 0.31$. La comparación directa ITS vs. tutoría humana arrojó $g = -0.11$ (no significativo).

\textbf{Kulik y Fletcher (2016)} \cite{kulik2016} analizaron 50 evaluaciones controladas: tamaño de efecto mediano $d = 0.66$; 92\% de evaluaciones mostraron superioridad sobre instrucción convencional.

\textbf{Ma et al. (2014)} \cite{ma2014} identificaron el \textbf{diagnóstico cognitivo en tiempo real} y la \textbf{remediación adaptativa} como los elementos más críticos para la efectividad de los ITS.

Un hallazgo crucial de VanLehn \cite{vanlehn2011} fue que los tutores \textit{step-based} (que verifican comprensión en cada paso) son significativamente más efectivos que los \textit{answer-based} (que solo evalúan respuestas finales). Este resultado fundamenta el diseño del modelo Bloom 8-Step Interactive de \tdl{}.

\subsection{Herramientas de Autoría}

El desarrollo de ITS históricamente ha requerido entre 200-300 horas de desarrollo por hora de instrucción \cite{murray1999}. Las herramientas de autoría buscan reducir esta barrera.

\textbf{CTAT} \cite{aleven2009} introdujo los \textit{Example-Tracing Tutors}, que pueden construirse ``enteramente sin programación'' mediante programación por demostración. Los autores demuestran comportamientos deseados, y el sistema generaliza las reglas.

\textbf{GIFT} \cite{sottilare2012} implementa una arquitectura modular con separación explícita entre módulo pedagógico y módulo de dominio. Esta separación permite reutilizar estrategias instruccionales entre diferentes dominios de conocimiento.

\textbf{AutoTutor} \cite{graesser2004} pioneró el uso de diálogo en lenguaje natural para tutoría, con tamaños de efecto reportados de 0.4 a 1.5. Su arquitectura incluye un Curriculum Script que organiza las preguntas y un Dialog Advancer que gestiona la conversación.

Sin embargo, incluso estas herramientas requieren conocimientos técnicos significativos y ecosistemas de software específicos que limitan su adopción fuera de entornos de investigación.

\subsection{La Transición a LLMs}

La investigación documenta problemas sistemáticos con LLM como tutores. Tack y Piech \cite{tack2022} encontraron que ``los LLM actuales no son buenos tutores por defecto''. Borchers et al. \cite{borchers2025} confirmaron que GPT-4 ``proporciona retroalimentación excesivamente directa''.

Sin embargo, evidencia temprana es prometedora: Pardos y Bhandari \cite{pardos2024} encontraron que hints generados por ChatGPT produjeron 17\% de ganancia de aprendizaje vs. 11.62\% para hints de tutores humanos---sin diferencia significativa. Kestin et al. \cite{kestin2025} reportaron que estudiantes con tutor de IA superaron a aquellos con instrucción activa presencial.

\subsection{Lecciones de IMS Learning Design}

Educational Modeling Language (EML), desarrollado por Rob Koper en Open University of the Netherlands, fue la base de IMS Learning Design (IMS LD v1.0, febrero 2003) \cite{koper2005}. Van Es y Koper demostraron que 16 planes de lección de tradiciones pedagógicas diversas podían codificarse exitosamente en IMS LD.

Sin embargo, Derntl et al. \cite{derntl2012} reportaron: ``IMS LD ha estado disponible desde 2003, y sin embargo no ha sido ampliamente adoptado''. Las causas identificadas fueron:

\begin{itemize}
    \item \textbf{Ecosistema inmaduro}: Griffiths et al. \cite{griffiths2005} encontraron que ``el round-tripping entre herramientas no es posible''.
    \item \textbf{Alto esfuerzo}: Berggren et al. \cite{berggren2005} documentaron relación 3:1 entre preparación y uso.
    \item \textbf{Desajuste terminológico}: Neumann y Oberhuemer \cite{neumann2008} identificaron que ``los conceptos del lenguaje difieren de los que un maestro usa para planificar''.
\end{itemize}

Crucialmente, Derntl et al. \cite{derntl2012} encontraron que tras 45 minutos de introducción, 78\% de profesores logró conformidad con soluciones expertas. ``La estructura conceptual de IMS LD \textbf{no} impide su uso para autoría''. Las barreras fueron de ecosistema, no conceptuales.

\subsection{Lecciones para TDL}

La historia de IMS LD sugiere cuatro lecciones para el diseño de \tdl{}:

\begin{enumerate}
    \item La simplicidad conceptual no garantiza adopción---el ecosistema importa.
    \item El esfuerzo debe ser proporcional al beneficio percibido.
    \item La terminología debe alinearse con el lenguaje que usan los docentes.
    \item La portabilidad reduce dependencia del ecosistema específico.
\end{enumerate}

\tdl{} intenta mitigar estos riesgos: no requiere herramientas especiales (cualquier editor de texto sirve), usa terminología familiar (eventos, unidades, prompts), y funciona en múltiples plataformas LLM comerciales. Sin embargo, hereda el riesgo fundamental de que el esfuerzo de formalización no se perciba proporcional al beneficio.
