% ============================================================================
% IX. CONCLUSIONES Y TRABAJO FUTURO
% ============================================================================

\section{Conclusiones y Trabajo Futuro}

\subsection{Resumen de Contribuciones}

Este artículo ha presentado \tdl{} (Tutor Description Language), una evolución de \adl{} que implementa una arquitectura de cuatro capas para sistemas de tutoría basados en LLM. Las contribuciones principales son:

\begin{enumerate}
    \item \textbf{Arquitectura desacoplada}: Separación de Engine, Modelo Instruccional, Secuencia de Aprendizaje y Contenido Fuente, permitiendo evolución y reutilización independientes.
    
    \item \textbf{Modelo instruccional como componente explícito}: Formalización del ``cómo enseñar'' como secuencia de eventos instruccionales reutilizables, alineados con teorías de Gagné y Bloom.
    
    \item \textbf{Separación metodología-contenido}: El mismo modelo instruccional puede aplicarse a cursos de diferentes disciplinas; el mismo contenido puede enseñarse con diferentes metodologías.
    
    \item \textbf{Evolución documentada desde ADL}: Análisis de las limitaciones de \adl{} para tutorías y principios de diseño que guiaron \tdl{}.
    
    \item \textbf{Especificación formal y herramientas}: Esquemas JSON Schema y validador Python para garantizar correctitud antes del despliegue.
    
    \item \textbf{Portabilidad multiplataforma}: Los mismos archivos \tdl{} funcionan en ChatGPT, Claude, Gemini y OpenWebUI sin modificación.
\end{enumerate}

\tdl{} no pretende ser conceptualmente novedoso, sino pragmáticamente útil: una implementación accesible de principios establecidos, adaptada a la realidad de los LLM actuales.

\subsection{Agenda de Investigación}

La validación empírica de \tdl{} requiere responder preguntas de investigación específicas:

\begin{itemize}
    \item \textbf{RQ1}: ¿Producen los tutores \tdl{} mayor ganancia de aprendizaje que LLM sin estructura pedagógica formal?
    \item \textbf{RQ2}: ¿Reduce \tdl{} la carga docente manteniendo o mejorando la calidad de la atención individualizada?
    \item \textbf{RQ3}: ¿Pueden docentes sin formación técnica avanzada crear tutores \tdl{} funcionales en tiempo razonable?
    \item \textbf{RQ4}: ¿Qué modelos instruccionales son más efectivos para qué tipos de contenido y población estudiantil?
\end{itemize}

\subsection{Hipótesis}

Basándonos en la literatura revisada, proponemos las siguientes hipótesis:

\begin{itemize}
    \item \textbf{H1}: Tutores \tdl{} con modelo Bloom 8-Step producirán mayor ganancia de aprendizaje que LLM con prompt genérico, en contenido técnico.
    
    \item \textbf{H2}: Docentes con tutores \tdl{} desplegados reportarán menor tiempo dedicado a atención individualizada repetitiva.
    
    \item \textbf{H3}: Docentes sin experiencia previa lograrán crear tutores \tdl{} funcionales en menos de 2 horas de trabajo.
    
    \item \textbf{H4}: Los modelos interactivos serán preferidos para contenido conceptual; los expositivos para contenido normativo.
\end{itemize}

\subsection{Diseño Experimental Propuesto}

Planificamos un estudio cuasi-experimental durante el curso 2025-2026 con tres condiciones:

\begin{enumerate}
    \item \textbf{Grupo TDL}: Estudiantes con acceso a tutor \tdl{} con modelo Bloom 8-Step.
    \item \textbf{Grupo LLM genérico}: Estudiantes con acceso a ChatGPT/Claude con prompt básico.
    \item \textbf{Grupo control}: Estudiantes con materiales tradicionales (apuntes, vídeos).
\end{enumerate}

Métricas:

\begin{itemize}
    \item Ganancia normalizada pre-post en tests de conocimiento.
    \item Tiempo de estudio auto-reportado.
    \item Engagement (número y profundidad de interacciones con el tutor).
    \item Satisfacción estudiantil (escala Likert).
    \item Carga docente (horas dedicadas a atención individualizada).
\end{itemize}

\subsection{Trabajo Futuro}

Identificamos varias direcciones de trabajo futuro:

\textbf{Editor visual (TDL Maker)}: Desarrollo de una herramienta web que permita diseñar secuencias de aprendizaje visualmente, generando el \yamlformat{} automáticamente. Esto eliminaría la barrera sintáctica para docentes sin experiencia técnica.

\textbf{Learning analytics}: Integración de hooks para registrar qué eventos se ejecutan, tiempos de respuesta del estudiante, patrones de interacción, y puntos de abandono. Estos datos informarían la mejora iterativa de modelos y secuencias.

\textbf{Repositorio comunitario}: Creación de un repositorio abierto de modelos instruccionales validados, donde diseñadores puedan compartir metodologías y docentes puedan descubrir modelos apropiados para sus necesidades.

\textbf{Extensiones para evaluación}: Añadir secciones para rúbricas de evaluación, criterios de calificación, y generación automática de informes de progreso estudiantil.

\textbf{Student model ligero}: Explorar la viabilidad de incorporar un modelo del estudiante simplificado, aprovechando la memoria conversacional de los LLM o almacenamiento externo, para habilitar cierta adaptación basada en historial.

\textbf{Integración con LMS}: Desarrollar conectores para integrar tutores \tdl{} con sistemas de gestión del aprendizaje (Moodle, Canvas), permitiendo autenticación, seguimiento de progreso, y sincronización de calificaciones.

\subsection{Conclusión Final}

La promesa de escalar la educación personalizada mediante IA no se cumplirá solo con modelos de lenguaje más potentes. Requiere métodos para que los educadores transfieran su expertise pedagógica a estos sistemas. \tdl{} ofrece un camino pragmático hacia este objetivo: permite que el ``cómo enseñar'' se formalice como conocimiento reutilizable mientras el ``qué enseñar'' permanece bajo control del docente.

Al separar responsabilidades en capas bien definidas y alinearse con teorías establecidas de diseño instruccional, \tdl{} facilita la colaboración entre diseñadores instruccionales, docentes y expertos en contenido. El resultado es un marco que amplifica el impacto de los educadores sin comprometer su autoría pedagógica.

La validación empírica determinará si las lecciones del pasado (particularmente de IMS Learning Design) han sido aprendidas. \tdl{} es una propuesta en esa dirección, sujeta a escrutinio experimental y mejora iterativa.
