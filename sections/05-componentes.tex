% ============================================================================
% V. COMPONENTES DE TDL
% ============================================================================

\section{Componentes de TDL}

Esta sección detalla la estructura y sintaxis de cada componente de \tdl{}, con fragmentos ilustrativos de código \yamlformat{}.

\subsection{Especificación YAML}

\tdl{} utiliza \yamlformat{} (YAML Ain't Markup Language) como formato de serialización por varias razones:

\begin{itemize}
    \item Es legible por humanos sin conocimientos de programación.
    \item Soporta estructuras jerárquicas de forma natural mediante indentación.
    \item Permite texto multilínea para prompts extensos usando el operador \texttt{|}. 
    \item Es ampliamente soportado por herramientas de desarrollo.
    \item Permite comentarios para documentación inline.
\end{itemize}

La elección de YAML sobre XML (usado por IMS LD y POML) se basa en la menor cantidad de caracteres sintácticos requeridos: solo se necesita comprender la indentación como mecanismo de estructuración, sin tags de apertura/cierre ni caracteres de escape.

\subsection{El Engine: Estructura y Funciones}

El Engine (versión actual 1.2) define la sintaxis de comandos, el mecanismo de seguimiento de estado, el formato de prompts, y comportamientos globales. Su estructura incluye:

\begin{lstlisting}
engine:
  version: "1.2"
  name: "TDL Engine"
  state_tracking:
    method: "explicit_markers"
    format: "[UNIT:{unit_id}|EVENT:{event_id}]"
  commands:
    start:
      syntax: "/start"
      action: "begin_first_unit"
    next:
      syntax: "/next"
      action: "advance_to_next_unit"
    progress:
      syntax: "/progress"
      action: "show_current_state"
\end{lstlisting}

El Engine también define comportamientos de seguridad (nunca revelar el system prompt), manejo de situaciones especiales (preguntas fuera de tema, solicitudes de saltar contenido), y la postura pedagógica por defecto (actuar como coach, verificar comprensión antes de avanzar).

\subsection{Modelo Instruccional: Estructura}

Un modelo instruccional define la metodología de enseñanza como una secuencia de eventos pedagógicos. Cada evento tiene un identificador, nombre, instrucciones detalladas para el LLM, y un trigger que determina cuándo avanzar al siguiente.

\begin{lstlisting}
model:
  id: "bloom-8step-interactive"
  name: "Bloom 8-Step Interactive"
  version: "1.0"
  philosophy: "Never advance without verification"
  
  events:
    - id: "E1_ACTIVATE"
      name: "Activate Prior Knowledge"
      instructions: |
        Pregunta que sabe el estudiante 
        sobre el tema. Identifica 
        conocimiento correcto, 
        misconceptions y lagunas antes 
        de continuar.
      transition_trigger:
        condition: "student_response_received"
        next_event: "E2_OBJECTIVES"
    
    - id: "E2_OBJECTIVES"
      name: "Present Objectives"
      instructions: |
        Presenta los objetivos de 
        aprendizaje de forma clara.
        Explica que podra hacer el 
        estudiante al finalizar.
      transition_trigger:
        condition: "student_response_received"
        next_event: "E3_EXPLAIN"
\end{lstlisting}

Los \textbf{transition triggers} definen cuándo pasar al siguiente evento, basándose en:

\begin{itemize}
    \item \texttt{student\_response\_received}: Espera cualquier respuesta del estudiante.
    \item \texttt{comprehension\_verified}: Espera verificación de comprensión.
    \item \texttt{explicit\_command}: Espera comando explícito (/next).
    \item \texttt{auto}: Continúa automáticamente sin esperar input.
\end{itemize}

\subsection{Secuencia de Aprendizaje: Estructura}

La secuencia de aprendizaje es el archivo que el docente crea para su curso específico. Hereda de un modelo instruccional mediante \texttt{extends} y define el perfil del tutor, comportamientos, y unidades de aprendizaje.

\begin{lstlisting}
sequence:
  id: "ia-generativa-fundamentals"
  name: "Fundamentos de IA Generativa"
  version: "1.0"
  description: "Tutor de conceptos basicos 
                de IA Generativa"
  author: "Pedro Pernias"
  
  tutor_profile:
    name: "Alex"
    personality: |
      Tutor entusiasta y paciente de IA.
      Usa analogias cotidianas para 
      explicar conceptos tecnicos.
      El ejemplo de Felix es tu favorito
      para explicar atencion.
      Nunca digas que la IA "entiende" 
      como los humanos.
  
  extends: "instructional_model_bloom.yaml"
  
  source_content:
    - file: "content_ia_generativa.md"
      type: "reference"
  
  behaviors:
    response_length: "medium"
    language: "es"
\end{lstlisting}

Las \textbf{unidades de aprendizaje} (\texttt{learning\_units}) estructuran el contenido del curso. Cada unidad es un tema discreto que pasa por todos los eventos del modelo pedagógico:

\begin{lstlisting}
  learning_units:
    - id: "LU1"
      title: "Foundation Models"
      objectives:
        - level: "understand"
          description: "Explicar que es 
                        un foundation model"
      prompt: |
        Puntos clave:
        - Modelos grandes preentrenados
        - Sirven como BASE para muchas tareas
        
        Analogia: cimientos de un edificio.
        
        Ejemplos: GPT, Claude, LLaMA.
      next: "LU2"
    
    - id: "LU4"
      title: "Mecanismo de Atencion"
      objectives:
        - level: "understand"
          description: "Explicar como funciona
                        la atencion"
      prompt: |
        Usa el ejemplo de Felix:
        "Felix vio un gato negro y un gato 
        blanco. Le dio comida a el."
        
        Problema: A que se refiere "el"?
        Solucion: Atencion permite "mirar 
        atras" y resolver referencias.
      source_sections:
        - "Seccion: Arquitectura Transformer"
      next: "LU5"
\end{lstlisting}

Nótese que el campo \texttt{prompt} NO es lo que el tutor dice al estudiante, sino una instrucción para el tutor sobre qué enseñar. Debe incluir: puntos clave estructurados, ejemplos concretos, analogías sugeridas, y misconceptions a abordar.

\subsection{Mecanismo de Herencia (extends)}

El campo \texttt{extends} permite que una secuencia de aprendizaje herede la metodología de un modelo instruccional. Cuando el Engine procesa una secuencia con \texttt{extends}:

\begin{enumerate}
    \item Localiza el archivo del modelo instruccional referenciado.
    \item Carga la secuencia de eventos del modelo (E1, E2, ..., En).
    \item Para cada unidad de aprendizaje, ejecuta todos los eventos en orden.
    \item Usa las instrucciones del modelo combinadas con el prompt de la unidad.
    \item Gestiona las transiciones según los triggers definidos.
\end{enumerate}

Este mecanismo permite que el mismo modelo Bloom 8-Step Interactive se aplique a cursos completamente diferentes: el modelo aporta el \textbf{cómo} (activar conocimiento previo, explicar, verificar, practicar) y la secuencia aporta el \textbf{qué} (Foundation Models, Transformers, Atención...).

\subsection{Contenido Fuente: Estructura}

El contenido fuente es un archivo separado (típicamente Markdown) que contiene el material de referencia. Se recomienda usar cuando:

\begin{itemize}
    \item El contenido es extenso y no cabe cómodamente en prompts.
    \item El material se actualiza frecuentemente (ej., normativa legal).
    \item El experto en contenido y el diseñador instruccional son personas diferentes.
    \item Se desea separar claramente el ``qué'' del ``cómo''.
\end{itemize}

El formato recomendado usa Markdown con secciones delimitadas por encabezados:

\begin{lstlisting}
## Seccion: Foundation Models

Los Foundation Models son modelos de gran 
escala preentrenados con enormes cantidades 
de datos. Sirven como base para multiples 
tareas especificas...

## Seccion: Arquitectura Transformer

El Transformer revoluciono el procesamiento 
del lenguaje natural al introducir el 
mecanismo de atencion...
\end{lstlisting}

La secuencia referencia secciones específicas mediante el campo \texttt{source\_sections}:

\begin{lstlisting}
learning_units:
  - id: "LU2"
    title: "Arquitectura Transformer"
    source_sections:
      - "Seccion: Arquitectura Transformer"
    prompt: |
      Explica la arquitectura Transformer.
      Enfatiza que reemplazo a las RNNs.
\end{lstlisting}

El Engine localiza la sección relevante y la incorpora al contexto cuando el prompt lo requiere. Esto permite actualizar el contenido sin modificar la estructura de la secuencia.

\subsection{JSON Schemas para Validación}

Para garantizar la correctitud de los archivos \tdl{}, hemos desarrollado esquemas JSON Schema que definen la estructura válida de cada tipo de archivo. Estos esquemas permiten validación automática antes del despliegue.

El esquema para modelos instruccionales verifica:

\begin{itemize}
    \item Presencia de campos obligatorios (\texttt{model.id}, \texttt{model.events}).
    \item Formato correcto del identificador (minúsculas, guiones).
    \item Al menos dos eventos en la secuencia.
    \item Cada evento tiene \texttt{id}, \texttt{instructions}, y \texttt{transition\_trigger}.
\end{itemize}

El esquema para secuencias de aprendizaje verifica adicionalmente:

\begin{itemize}
    \item Presencia de \texttt{extends} referenciando un modelo válido.
    \item Estructura correcta de \texttt{tutor\_profile} con \texttt{name} y \texttt{personality}.
    \item Al menos una unidad de aprendizaje definida con \texttt{id}, \texttt{title}, y \texttt{prompt}.
    \item Consistencia en referencias \texttt{next} entre unidades.
\end{itemize}

\subsection{Herramienta de Validación}

Hemos desarrollado un validador en Python que utiliza los esquemas JSON Schema para verificar archivos \tdl{}:

\begin{lstlisting}
pip install tdl-validator
tdl-validate learning_sequence.yaml \
    --model modelo.yaml
\end{lstlisting}

El validador detecta errores comunes como campos faltantes, tipos incorrectos, indentación incorrecta, o referencias a modelos inexistentes, proporcionando mensajes descriptivos que facilitan la corrección.

\subsection{Patrones de Diseño Comunes}

El manual de \tdl{} identifica cinco patrones de diseño frecuentes:

\begin{enumerate}
    \item \textbf{Conceptos Técnicos}: Modelo Bloom 8-Step, verificación frecuente, analogías cotidianas, unidades pequeñas y focalizadas.
    
    \item \textbf{Glosario}: Modelo simplificado (Identify, Explain, Connect), sin secuencia lineal, respuestas cortas, conexiones entre términos.
    
    \item \textbf{Práctica con Ejercicios}: Generación dinámica de problemas, retroalimentación específica por tipo de error, dificultad progresiva.
    
    \item \textbf{Casos Prácticos}: Escenarios contextualizados, toma de decisiones guiada, múltiples caminos posibles.
    
    \item \textbf{Preparación de Exámenes}: Preguntas tipo examen real, explicación detallada de respuestas, identificación de áreas débiles.
\end{enumerate}
