% ============================================================================
% VIII. DISCUSIÓN
% ============================================================================

\section{Discusión}

\subsection{¿Qué Aporta TDL?}

\tdl{} no introduce conceptos pedagógicos novedosos. Su aportación es de \textbf{implementación pragmática}: traduce principios establecidos de diseño instruccional a un formato que funciona sobre LLM comerciales sin requerir infraestructura adicional.

Específicamente, \tdl{} ofrece:

\begin{itemize}
    \item \textbf{Portabilidad inmediata}: Los mismos archivos funcionan en ChatGPT, Claude, Gemini y OpenWebUI.
    
    \item \textbf{Formato accesible}: YAML es más legible que XML para usuarios no técnicos.
    
    \item \textbf{Reutilización demostrada}: Un modelo instruccional puede aplicarse a múltiples cursos de diferentes disciplinas.
    
    \item \textbf{Especificación formal}: JSON Schemas permiten validación automática.
    
    \item \textbf{Alineamiento con teoría}: Los eventos instruccionales corresponden a marcos establecidos (Gagné, Bloom).
\end{itemize}

\subsection{Comparativa con Trabajos Relacionados}

En relación con otros lenguajes de estructuración de prompts:

\textbf{PDL (IBM)}: Ofrece capacidades técnicas sofisticadas (sistema de tipos, llamadas a funciones) pero está orientado a desarrolladores, no a educadores. No incorpora conceptos de diseño instruccional.

\textbf{POML (Microsoft)}: Su sintaxis XML proporciona gran expresividad pero introduce complejidad innecesaria para el caso de uso educativo. El sistema de estilos CSS es potente pero alejado del lenguaje de los docentes.

\textbf{IMS Learning Design}: Conceptualmente completo pero fracasó en adopción por problemas de ecosistema, no de complejidad conceptual. \tdl{} intenta aprender de este fracaso priorizando simplicidad y portabilidad.

\tdl{} ocupa un nicho específico: combina la legibilidad de \yamlformat{} con conceptos propios del diseño instruccional (eventos, modelos pedagógicos, secuencias de aprendizaje) que ni PDL ni POML contemplan.

\subsection{Implicaciones para el Diseño Instruccional}

La arquitectura de \tdl{} sugiere una distribución natural de roles profesionales:

\begin{itemize}
    \item \textbf{Diseñador instruccional senior}: Crea y valida modelos instruccionales reutilizables. Requiere conocimiento profundo de teorías de aprendizaje.
    
    \item \textbf{Docente / Diseñador de curso}: Crea secuencias de aprendizaje aplicando modelos existentes. Requiere conocimiento del contenido y familiaridad básica con \yamlformat{}.
    
    \item \textbf{Experto en contenido}: Crea y actualiza el contenido fuente. Solo requiere dominio de la materia.
\end{itemize}

Esta separación permite colaboración eficiente: un diseñador instruccional puede crear un modelo ``Aprendizaje Basado en Problemas'' que luego usen docentes de Medicina, Ingeniería y Derecho, cada uno con su contenido específico.

\subsection{Consideraciones Éticas}

\tdl{} mantiene al docente como autor y responsable último del proceso educativo. El LLM ejecuta lo que el docente ha especificado, preservando la autoría pedagógica. La transparencia de \yamlformat{} permite que cualquier observador inspeccione exactamente qué metodología sigue el tutor.

Sin embargo, persisten preocupaciones:

\begin{itemize}
    \item \textbf{Sesgos del LLM}: El modelo subyacente puede introducir sesgos no especificados en \tdl{}.
    
    \item \textbf{Dependencia tecnológica}: Los estudiantes pueden desarrollar expectativas de disponibilidad continua que los tutores humanos no pueden satisfacer.
    
    \item \textbf{Privacidad}: Las conversaciones con el tutor se procesan en servidores de terceros (OpenAI, Anthropic, Google).
\end{itemize}

\subsection{Limitaciones}

Reconocemos varias limitaciones de \tdl{} en su estado actual:

\textbf{Dependencia del LLM subyacente}: \tdl{} asume que el LLM seguirá fielmente las instrucciones del Engine y los prompts. En la práctica, los modelos pueden desviarse ocasionalmente, especialmente en conversaciones largas.

\textbf{Curva de aprendizaje para YAML}: Aunque \yamlformat{} es más legible que otros formatos, sigue requiriendo atención a la indentación y sintaxis. Un editor visual que genere \yamlformat{} automáticamente facilitaría la adopción.

\textbf{Pedagogías improvisadas}: \tdl{} está optimizado para metodologías estructuradas con secuencias predefinidas. Enfoques altamente improvisados o emergentes pueden no beneficiarse tanto de la formalización.

\textbf{Ausencia de evaluación empírica}: Este artículo presenta \tdl{} como arquitectura y especificación. La validación empírica con estudiantes reales está planificada pero aún no ejecutada.

\textbf{Ausencia de student model}: \tdl{} no implementa un modelo del estudiante. No hay diagnóstico cognitivo, seguimiento de misconceptions, ni adaptación basada en el historial del aprendiz.

\subsection{Riesgos Heredados de IMS LD}

La historia de IMS Learning Design sugiere cautela. A pesar de su solidez conceptual, IMS LD no logró adopción generalizada. \tdl{} mitiga algunos de los riesgos identificados:

\begin{itemize}
    \item \textbf{Ecosistema}: \tdl{} no requiere herramientas especiales; cualquier editor de texto sirve.
    \item \textbf{Portabilidad}: \tdl{} funciona en múltiples plataformas sin modificación.
    \item \textbf{Terminología}: \tdl{} usa términos familiares (eventos, prompts, unidades).
\end{itemize}

Sin embargo, \tdl{} hereda el riesgo fundamental de que el esfuerzo de formalización no se perciba proporcional al beneficio obtenido. Solo la validación empírica determinará si los docentes adoptan \tdl{} en la práctica.

\subsection{Lo Que TDL No Es}

Es importante reiterar lo que \tdl{} no pretende ser:

\begin{itemize}
    \item \tdl{} \textbf{no es un ITS completo}: Carece de los componentes de diagnóstico y adaptación que definen a los ITS tradicionales.
    
    \item \tdl{} \textbf{no garantiza efectividad}: Un tutor \tdl{} mal diseñado será tan inefectivo como cualquier otra instrucción mal diseñada.
    
    \item \tdl{} \textbf{no reemplaza al docente}: El docente sigue siendo el autor del diseño pedagógico; \tdl{} solo formaliza y escala su expertise.
    
    \item \tdl{} \textbf{no resuelve limitaciones de los LLM}: Si el modelo subyacente no puede seguir instrucciones complejas, \tdl{} no puede compensarlo.
\end{itemize}
