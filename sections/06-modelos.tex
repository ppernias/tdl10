% ============================================================================
% VI. MODELOS INSTRUCCIONALES
% ============================================================================

\section{Modelos Instruccionales en TDL}

El concepto de modelo instruccional es central en \tdl{}. Esta sección profundiza en su fundamentación teórica y presenta los dos modelos de referencia incluidos en la especificación.

\subsection{El Concepto de Evento Instruccional}

Los eventos instruccionales en \tdl{} se inspiran en la teoría de Robert Gagné \cite{gagne1985}, quien identificó nueve eventos que optimizan el aprendizaje. \tdl{} adapta este marco al contexto de tutorías conversacionales, donde cada evento representa una fase con propósito pedagógico específico.

La Tabla~\ref{tab:gagne_mapping} muestra la correspondencia entre los eventos de Gagné y los eventos típicos de \tdl{}.

\begin{table}[htbp]
\caption{Correspondencia entre Eventos de Gagné y TDL}
\label{tab:gagne_mapping}
\centering
\begin{tabular}{p{3.2cm}p{3.5cm}}
\toprule
\textbf{Evento de Gagné} & \textbf{Evento TDL} \\
\midrule
Ganar atención & Attention / Activate \\
Informar objetivos & Objectives \\
Estimular recuerdo & Recall / Activate \\
Presentar contenido & Explain / Present \\
Proporcionar guía & Guidance / Elaborate \\
Elicitar desempeño & Practice / Elicit \\
Dar retroalimentación & Verify / Feedback \\
Evaluar desempeño & Assess \\
Mejorar retención & Summary / Transfer \\
\bottomrule
\end{tabular}
\end{table}

Esta correspondencia no es rígida: \tdl{} permite flexibilidad en cómo se organizan los eventos, siempre que mantengan coherencia pedagógica.

\subsection{Relación con Transaction Shells}

Los modelos instruccionales de \tdl{} son funcionalmente análogos a los \textit{transaction shells} de Merrill \cite{merrill1991}: algoritmos pedagógicos reutilizables para diferentes contenidos. Un transaction shell especifica la secuencia de interacciones sin conocer el contenido específico; el modelo instruccional de \tdl{} hace exactamente lo mismo.

Esta analogía refuerza que \tdl{} no introduce conceptos nuevos, sino que implementa principios establecidos en un contexto tecnológico nuevo (LLMs).

\subsection{Modelo de Referencia: Bloom 8-Step Interactive}

El modelo Bloom 8-Step Interactive implementa un enfoque socrático donde el tutor verifica constantemente la comprensión antes de avanzar. Su filosofía central es: \textbf{``Never advance without verification''} (nunca avanzar sin verificar).

Este diseño se alinea con la evidencia de VanLehn \cite{vanlehn2011}: los tutores \textit{step-based} ($d = 0.76$) superan significativamente a los \textit{answer-based} ($d = 0.31$).

A pesar del nombre ``8-Step'', el modelo implementa 10 eventos para mayor granularidad en la verificación:

\begin{table}[htbp]
\caption{Eventos del Modelo Bloom 8-Step Interactive}
\label{tab:bloom_events}
\centering
\begin{tabular}{clp{2.8cm}}
\toprule
\textbf{ID} & \textbf{Evento} & \textbf{Propósito} \\
\midrule
E1 & ACTIVATE & Activar conocimiento previo \\
E2 & OBJECTIVES & Presentar objetivos \\
E3 & EXPLAIN & Presentar contenido \\
E4 & VERIFY\_EXPLAIN & Verificar comprensión inicial \\
E5 & ELABORATE & Profundizar si correcto \\
E6 & VERIFY\_ELABORATE & Verificar profundización \\
E7 & PRACTICE & Práctica guiada \\
E8 & VERIFY\_PRACTICE & Verificar práctica \\
E9 & SUMMARY & Resumen y conexiones \\
E10 & CLOSE & Cierre y transición \\
\bottomrule
\end{tabular}
\end{table}

\subsubsection{Características del Modelo}

\begin{itemize}
    \item \textbf{Verificación continua}: Cada explicación (E3, E5) va seguida de verificación (E4, E6).
    \item \textbf{Feedback inmediato}: El tutor evalúa cada respuesta antes de continuar.
    \item \textbf{Progresión cognitiva}: De recuerdo a comprensión, de comprensión a aplicación, de aplicación a síntesis (alineado con taxonomía de Bloom \cite{anderson2001}).
    \item \textbf{Detección de misconceptions}: E1 identifica conocimiento incorrecto antes de enseñar.
\end{itemize}

\subsubsection{Casos de Uso Ideales}

Este modelo es ideal para:

\begin{itemize}
    \item Conceptos técnicos complejos donde cada concepto es prerequisito del siguiente.
    \item Dominios donde los errores conceptuales son costosos (medicina, ingeniería, derecho).
    \item Estudiantes que necesitan guía activa y verificación frecuente.
    \item Situaciones donde la comprensión profunda es más importante que la cobertura amplia.
\end{itemize}

\subsection{Modelo de Referencia: Expository}

El modelo Expository implementa un enfoque de clase magistral estructurada, basado directamente en los nueve eventos de Gagné. El tutor expone el contenido de forma organizada antes de solicitar participación activa.

Su filosofía es: \textbf{``Transmisión estructurada''} ---presentación organizada con verificaciones concentradas al final.

\begin{table}[htbp]
\caption{Eventos del Modelo Expository (basado en Gagné)}
\label{tab:expository_events}
\centering
\begin{tabular}{clp{3cm}}
\toprule
\textbf{ID} & \textbf{Evento} & \textbf{Propósito} \\
\midrule
E1 & ATTENTION & Captar atención inicial \\
E2 & OBJECTIVES & Informar objetivos \\
E3 & RECALL & Estimular recuerdo previo \\
E4 & PRESENT & Presentar contenido \\
E5 & GUIDANCE & Proporcionar guía \\
E6 & ELICIT & Elicitar desempeño \\
E7 & FEEDBACK & Dar retroalimentación \\
E8 & ASSESS & Evaluar desempeño \\
E9 & TRANSFER & Mejorar retención \\
\bottomrule
\end{tabular}
\end{table}

\subsubsection{Diferencias con el Modelo Interactivo}

La diferencia clave es que el tutor puede dar exposiciones más largas (5-8 párrafos) antes de pausar para interacción. El estudiante tiene un rol más receptivo inicialmente, con participación activa concentrada en E6-E8.

\subsubsection{Casos de Uso Ideales}

Este modelo es ideal para:

\begin{itemize}
    \item Contenido normativo, legal o regulatorio donde la precisión es crítica.
    \item Primera exposición a un tema completamente nuevo.
    \item Cobertura rápida de material introductorio o contextual.
    \item Estudiantes que prefieren recibir información completa antes de interactuar.
\end{itemize}

\subsection{Creación de Modelos Personalizados}

\tdl{} no limita a los docentes a los modelos de referencia. Un diseñador instruccional puede crear modelos personalizados siguiendo este proceso:

\begin{enumerate}
    \item Identificar la filosofía pedagógica guía.
    \item Dibujar la secuencia de eventos en papel.
    \item Definir qué ocurre en cada fase y su propósito.
    \item Definir cuándo se pasa a la siguiente (triggers).
    \item Escribir instrucciones detalladas para cada evento.
\end{enumerate}

Por ejemplo, el modelo 5E (Engage, Explore, Explain, Elaborate, Evaluate) para indagación científica:

\begin{lstlisting}
model:
  id: "5e-inquiry"
  name: "5E Inquiry Model"
  philosophy: "Discovery through guided 
               exploration"
  
  events:
    - id: "E1_ENGAGE"
      name: "Engage"
      instructions: |
        Captura atencion con pregunta
        provocadora o fenomeno intrigante.
        NO reveles la respuesta todavia.
      transition_trigger:
        condition: "student_response_received"
        next_event: "E2_EXPLORE"
    
    - id: "E2_EXPLORE"
      name: "Explore"
      instructions: |
        Guia al estudiante a descubrir
        por si mismo mediante preguntas
        orientadoras. NO des respuestas
        directas en esta fase.
      transition_trigger:
        condition: "exploration_complete"
        next_event: "E3_EXPLAIN"
\end{lstlisting}

La creación de modelos personalizados abre la posibilidad de un \textbf{repositorio comunitario} donde diseñadores instruccionales compartan metodologías validadas que otros docentes puedan aplicar directamente a sus cursos.

\subsection{Limitaciones de los Modelos}

Es importante reconocer las limitaciones de los modelos instruccionales en \tdl{}:

\begin{itemize}
    \item \textbf{Secuencia predefinida}: Los modelos definen secuencias que se repiten para cada unidad. No hay ramificación condicional basada en el desempeño del estudiante.
    \item \textbf{Sin diagnóstico cognitivo}: Los modelos no detectan misconceptions específicas más allá de lo que el LLM puede inferir conversacionalmente.
    \item \textbf{Dependencia del LLM}: La calidad de ejecución depende de cuán fielmente el LLM siga las instrucciones. Modelos más capaces (GPT-4, Claude 3) producen mejores resultados.
\end{itemize}

Estas limitaciones son deliberadas: mantienen \tdl{} simple y portable. Funcionalidades más avanzadas requerirían infraestructura que contradice el objetivo de despliegue sobre plataformas comerciales existentes.
