% ============================================================================
% VII. ESPECIFICACIÓN TÉCNICA Y PORTABILIDAD
% ============================================================================

\section{Especificación Técnica y Portabilidad}

Una ventaja clave de \tdl{} es su portabilidad: los mismos archivos funcionan en múltiples plataformas LLM sin modificación. Esta sección describe cómo se logra esta portabilidad.

\subsection{Portabilidad entre Plataformas}

\tdl{} ha sido probado en cuatro plataformas principales. La Tabla~\ref{tab:plataformas} muestra la configuración para cada una.

\begin{table}[htbp]
\caption{Configuración de TDL por Plataforma}
\label{tab:plataformas}
\centering
\begin{tabular}{lll}
\toprule
\textbf{Plataforma} & \textbf{Engine} & \textbf{Archivos TDL} \\
\midrule
ChatGPT GPTs & Instructions & Knowledge \\
Claude Projects & Project Instructions & Project Knowledge \\
Gemini Gems & Instructions & Attached Files \\
OpenWebUI & System Prompt & Knowledge + RAG \\
\bottomrule
\end{tabular}
\end{table}

\subsection{Proceso de Despliegue}

El proceso de despliegue es consistente entre plataformas:

\begin{enumerate}
    \item \textbf{Copiar el Engine}: El contenido del Engine se copia en el campo de instrucciones del sistema (Instructions, System Prompt, o equivalente).
    
    \item \textbf{Subir archivos TDL}: Los archivos \yamlformat{} (modelo instruccional y secuencia de aprendizaje) se suben al área de conocimiento de la plataforma.
    
    \item \textbf{Subir contenido fuente} (opcional): Si se usa contenido fuente separado, se sube también al área de conocimiento.
    
    \item \textbf{Configurar conversation starters}: Se configuran los mensajes iniciales sugeridos con ``Hola'' y ``/start''.
\end{enumerate}

\subsection{Configuración por Plataforma}

\subsubsection{ChatGPT GPTs}

En ChatGPT, los GPTs personalizados tienen un campo ``Instructions'' que acepta hasta 8,000 caracteres. El Engine se copia aquí. Los archivos \tdl{} y el contenido fuente se suben a ``Knowledge'', donde ChatGPT puede acceder a ellos mediante retrieval.

\begin{lstlisting}
# Configuracion ChatGPT GPT
Instructions: [Contenido del Engine]
Knowledge: 
  - instructional_model_bloom8.yaml
  - learning_sequence_curso.yaml
  - content_curso.md
Conversation Starters:
  - "Hola"
  - "/start"
\end{lstlisting}

\subsubsection{Claude Projects}

Claude ofrece ``Projects'' con instrucciones de proyecto y archivos de conocimiento. El Engine se copia en ``Project Instructions''. Los archivos \tdl{} se añaden como ``Project Knowledge''.

\subsubsection{Gemini Gems}

Google Gemini tiene ``Gems'' con un campo de instrucciones y archivos adjuntos. La configuración es análoga a las anteriores plataformas.

\subsubsection{OpenWebUI}

OpenWebUI es una interfaz open-source que puede conectarse a diferentes backends LLM. El Engine se configura como System Prompt, y los archivos \tdl{} se integran mediante el sistema RAG (Retrieval Augmented Generation) de la plataforma.

\subsection{Ventajas de la Portabilidad}

La portabilidad de \tdl{} ofrece varias ventajas:

\begin{itemize}
    \item \textbf{No vendor lock-in}: Las instituciones pueden migrar entre plataformas sin reescribir sus especificaciones.
    
    \item \textbf{Comparación de modelos}: El mismo tutor puede desplegarse en GPT-4, Claude y Gemini para comparar rendimiento.
    
    \item \textbf{Redundancia}: Si una plataforma tiene problemas de disponibilidad, el tutor puede desplegarse temporalmente en otra.
    
    \item \textbf{Elección por caso de uso}: Diferentes cursos pueden usar diferentes plataformas según sus necesidades (costo, capacidades específicas, políticas institucionales).
\end{itemize}

\subsection{Limitaciones de Portabilidad}

A pesar de la portabilidad general, existen algunas diferencias entre plataformas:

\begin{itemize}
    \item \textbf{Capacidad de seguimiento de instrucciones}: Modelos más avanzados (GPT-4, Claude 3) siguen las instrucciones del Engine más fielmente que modelos más pequeños.
    
    \item \textbf{Límites de contexto}: Algunas plataformas tienen límites más restrictivos en el tamaño del system prompt o el contenido de knowledge.
    
    \item \textbf{Retrieval quality}: La calidad de recuperación de información de los archivos de knowledge varía entre plataformas.
    
    \item \textbf{Funcionalidades específicas}: Algunas plataformas ofrecen capacidades adicionales (code interpreter, browsing) que \tdl{} no aprovecha actualmente.
\end{itemize}

\subsection{Recomendaciones de Despliegue}

Basándonos en nuestra experiencia, ofrecemos las siguientes recomendaciones:

\begin{enumerate}
    \item \textbf{Probar en múltiples plataformas}: Antes de despliegue en producción, verificar comportamiento en al menos dos plataformas.
    
    \item \textbf{Usar modelos de última generación}: Para tutorías complejas, usar GPT-4, Claude 3 Opus/Sonnet, o equivalentes.
    
    \item \textbf{Validar antes de subir}: Ejecutar el validador \tdl{} en todos los archivos antes del despliegue.
    
    \item \textbf{Documentar configuración}: Mantener un registro de qué versión de qué archivos están desplegados en cada plataforma.
\end{enumerate}

\subsection{Herramientas de Soporte}

El ecosistema \tdl{} incluye herramientas de soporte:

\begin{itemize}
    \item \textbf{Validador Python}: Verifica sintaxis y estructura de archivos \tdl{}.
    \item \textbf{Plantillas de referencia}: Modelos instruccionales y secuencias de ejemplo listos para adaptar.
    \item \textbf{Documentación}: Guía completa de la especificación y mejores prácticas.
\end{itemize}

Trabajo futuro incluye el desarrollo de un \textbf{TDL Maker}: una herramienta web visual que permita diseñar secuencias de aprendizaje mediante drag-and-drop, generando el \yamlformat{} automáticamente. Esta herramienta reduciría aún más la barrera de entrada para docentes sin familiaridad con formatos de texto estructurado.
