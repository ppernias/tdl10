% ============================================================================
% VI. INSTRUCTIONAL MODELS
% ============================================================================
\section{Instructional Models in TDL}

The concept of instructional model is central to TDL. This section elaborates on its theoretical foundation and presents the two reference models included in the specification.

\subsection{The Instructional Event Concept}

Instructional events in TDL are inspired by Robert Gagn\'{e}'s theory \cite{gagne1985}, who identified nine events that optimize learning. TDL adapts this framework to the context of conversational tutoring, where each event represents a phase with a specific pedagogical purpose.

Table~\ref{tab:gagne_mapping} shows the correspondence between Gagn\'{e}'s events and typical TDL events.

\begin{table}[htbp]
\caption{Correspondence Between Gagn\'{e}'s Events and TDL}
\label{tab:gagne_mapping}
\centering
\begin{tabular}{p{3.2cm}p{3.5cm}}
\toprule
\textbf{Gagn\'{e}'s Event} & \textbf{TDL Event} \\
\midrule
Gain attention & Attention / Activate \\
Inform objectives & Objectives \\
Stimulate recall & Recall / Activate \\
Present content & Explain / Present \\
Provide guidance & Guidance / Elaborate \\
Elicit performance & Practice / Elicit \\
Provide feedback & Verify / Feedback \\
Assess performance & Assess \\
Enhance retention & Summary / Transfer \\
\bottomrule
\end{tabular}
\end{table}

This correspondence is not rigid: TDL allows flexibility in how events are organized, as long as they maintain pedagogical coherence.

\subsection{Relationship with Transaction Shells}

TDL's instructional models are functionally analogous to Merrill's \textit{transaction shells} \cite{merrill1991}: reusable pedagogical algorithms for different contents. A transaction shell specifies the sequence of interactions without knowing the specific content; TDL's instructional model does exactly the same.

This analogy reinforces that TDL does not introduce new concepts, but implements established principles in a new technological context (LLMs).

\subsection{Reference Model: Bloom 8-Step Interactive}

The Bloom 8-Step Interactive model implements a Socratic approach where the tutor constantly verifies comprehension before advancing. Its central philosophy is: \textbf{``Never advance without verification.''}

This design aligns with VanLehn's evidence \cite{vanlehn2011}: \textit{step-based} tutors ($d = 0.76$) significantly outperform \textit{answer-based} tutors ($d = 0.31$).

Despite the ``8-Step'' name, the model implements 10 events for greater verification granularity:

\begin{table}[htbp]
\caption{Events of the Bloom 8-Step Interactive Model}
\label{tab:bloom_events}
\centering
\begin{tabular}{clp{2.8cm}}
\toprule
\textbf{ID} & \textbf{Event} & \textbf{Purpose} \\
\midrule
E1 & ACTIVATE & Activate prior knowledge \\
E2 & OBJECTIVES & Present objectives \\
E3 & EXPLAIN & Present content \\
E4 & VERIFY\_EXPLAIN & Verify initial comprehension \\
E5 & ELABORATE & Deepen if correct \\
E6 & VERIFY\_ELABORATE & Verify deepening \\
E7 & PRACTICE & Guided practice \\
E8 & VERIFY\_PRACTICE & Verify practice \\
E9 & SUMMARY & Summary and connections \\
E10 & CLOSE & Closure and transition \\
\bottomrule
\end{tabular}
\end{table}

\subsubsection{Model Characteristics}

\begin{itemize}
    \item \textbf{Continuous verification}: Each explanation (E3, E5) is followed by verification (E4, E6).
    \item \textbf{Immediate feedback}: The tutor evaluates each response before continuing.
    \item \textbf{Cognitive progression}: From recall to comprehension, from comprehension to application, from application to synthesis (aligned with Bloom's taxonomy \cite{anderson2001}).
    \item \textbf{Misconception detection}: E1 identifies incorrect knowledge before teaching.
\end{itemize}

\subsubsection{Ideal Use Cases}

This model is ideal for:

\begin{itemize}
    \item Complex technical concepts where each concept is a prerequisite for the next.
    \item Domains where conceptual errors are costly (medicine, engineering, law).
    \item Students who need active guidance and frequent verification.
    \item Situations where deep understanding is more important than broad coverage.
\end{itemize}

\subsection{Reference Model: Expository}

The Expository model implements a structured lecture approach, based directly on Gagn\'{e}'s nine events. The tutor presents content in an organized manner before requesting active participation.

Its philosophy is: \textbf{``Structured transmission''}---organized presentation with verifications concentrated at the end.

\begin{table}[htbp]
\caption{Events of the Expository Model (based on Gagn\'{e})}
\label{tab:expository_events}
\centering
\begin{tabular}{clp{3cm}}
\toprule
\textbf{ID} & \textbf{Event} & \textbf{Purpose} \\
\midrule
E1 & ATTENTION & Capture initial attention \\
E2 & OBJECTIVES & Inform objectives \\
E3 & RECALL & Stimulate prior recall \\
E4 & PRESENT & Present content \\
E5 & GUIDANCE & Provide guidance \\
E6 & ELICIT & Elicit performance \\
E7 & FEEDBACK & Provide feedback \\
E8 & ASSESS & Assess performance \\
E9 & TRANSFER & Enhance retention \\
\bottomrule
\end{tabular}
\end{table}

\subsubsection{Differences from the Interactive Model}

The key difference is that the tutor can give longer expositions (5--8 paragraphs) before pausing for interaction. The student has a more receptive role initially, with active participation concentrated in E6--E8.

\subsubsection{Ideal Use Cases}

This model is ideal for:

\begin{itemize}
    \item Normative, legal, or regulatory content where precision is critical.
    \item First exposure to a completely new topic.
    \item Rapid coverage of introductory or contextual material.
    \item Students who prefer receiving complete information before interacting.
\end{itemize}

\subsection{Creating Custom Models}

TDL does not limit teachers to reference models. An instructional designer can create custom models following this process:

\begin{enumerate}
    \item Identify the guiding pedagogical philosophy.
    \item Draw the event sequence on paper.
    \item Define what happens in each phase and its purpose.
    \item Define when to advance to the next (triggers).
    \item Write detailed instructions for each event.
\end{enumerate}

For example, the 5E model (Engage, Explore, Explain, Elaborate, Evaluate) for scientific inquiry:

\begin{lstlisting}
model:
  id: "5e-inquiry"
  name: "5E Inquiry Model"
  philosophy: "Discovery through guided
               exploration"

  events:
    - id: "E1_ENGAGE"
      name: "Engage"
      instructions: |
        Capture attention with a provocative
        question or intriguing phenomenon.
        Do NOT reveal the answer yet.
      transition_trigger:
        condition: "student_response_received"
        next_event: "E2_EXPLORE"

    - id: "E2_EXPLORE"
      name: "Explore"
      instructions: |
        Guide the student to discover on
        their own through guiding questions.
        Do NOT give direct answers in
        this phase.
      transition_trigger:
        condition: "exploration_complete"
        next_event: "E3_EXPLAIN"
\end{lstlisting}

Custom model creation opens the possibility of a \textbf{community repository} where instructional designers share validated methodologies that other teachers can apply directly to their courses.

\subsection{Model Limitations}

It is important to recognize the limitations of instructional models in TDL:

\begin{itemize}
    \item \textbf{Predefined sequence}: Models define sequences that repeat for each unit. There is no conditional branching based on student performance.
    \item \textbf{No cognitive diagnosis}: Models do not detect specific misconceptions beyond what the LLM can infer conversationally.
    \item \textbf{LLM dependence}: Execution quality depends on how faithfully the LLM follows instructions. More capable models (GPT-4, Claude 3) produce better results.
\end{itemize}

These limitations are deliberate: they keep TDL simple and portable. More advanced functionalities would require infrastructure that contradicts the goal of deployment on existing commercial platforms.
