\documentclass[conference]{IEEEtran}

% Paquetes basicos - ORDEN IMPORTANTE
\usepackage[utf8]{inputenc}
\usepackage[T1]{fontenc}
\usepackage[spanish,es-noshorthands]{babel}
\usepackage{cite}
\usepackage{amsmath,amssymb,amsfonts}
\usepackage{graphicx}
\usepackage{textcomp}
\usepackage{xcolor}
\usepackage{booktabs}
\usepackage{array}
\usepackage{multirow}
\usepackage{tikz}
\usepackage{float}
\usepackage{url}
\usepackage{hyperref}
\usepackage{listings}

% Definicion del lenguaje YAML para listings
\lstdefinelanguage{yaml}{
  morekeywords={true,false,null,y,n},
  sensitive=false,
  morecomment=[l]{\#\#},
  morestring=[b]",
  morestring=[b]',
  alsoletter={-},
  tabsize=2
}

\usetikzlibrary{shapes.geometric, arrows.meta, positioning}

% Colores para codigo
\definecolor{codegreen}{RGB}{34, 139, 34}
\definecolor{codegray}{RGB}{128, 128, 128}
\definecolor{codeblue}{RGB}{0, 0, 180}
\definecolor{backcolour}{RGB}{248, 248, 248}

% Configuracion de listings para YAML
\lstset{
    basicstyle=\footnotesize\ttfamily,
    backgroundcolor=\color{backcolour},
    frame=single,
    framerule=0.5pt,
    breaklines=true,
    breakatwhitespace=true,
    tabsize=2,
    showstringspaces=false,
    numbers=none,
    xleftmargin=2pt,
    xrightmargin=2pt,
    aboveskip=6pt,
    belowskip=6pt,
    keywordstyle=\color{codeblue}\bfseries,
    commentstyle=\color{codegray}\ttfamily,
    stringstyle=\color{codegreen}
}

% Comandos personalizados
\newcommand{\tdl}{\textsc{tdl}}
\newcommand{\adl}{\textsc{adl}}
\newcommand{\yamlformat}{\textsc{yaml}}

\begin{document}

\title{Del ADL al TDL: Una Arquitectura Desacoplada para Tutores Basados en LLM Fundamentada en Principios de Diseño Instruccional}

\author{
    \IEEEauthorblockN{Pedro A. Pernías Peco}
    \IEEEauthorblockA{
        Departamento de Lenguajes y Sistemas Informáticos\\
        Universidad de Alicante\\
        Alicante, España\\
        p.pernias@ua.es
    }
    \and
    \IEEEauthorblockN{M. Pilar Escobar Esteban}
    \IEEEauthorblockA{
        Departamento de Lenguajes y Sistemas Informáticos\\
        Universidad de Alicante\\
        Alicante, España\\
        pilar.escobar@ua.es
    }
}

\maketitle

% ============================================================================
% ABSTRACT
% ============================================================================

\begin{abstract}
Los Modelos de Lenguaje de Gran Escala (LLM) ofrecen una oportunidad sin precedentes para escalar la tutoría personalizada, pero la investigación documenta que carecen de alineamiento pedagógico: proporcionan respuestas directas en lugar de guiar el aprendizaje. La separación entre metodología instruccional y contenido es un principio establecido en diseño instruccional desde la Component Display Theory (Merrill, 1983), implementado en sistemas como IMS Learning Design (2003) y frameworks como GIFT. Trabajos previos como el Assistant Description Language (\adl{}) permiten a los docentes codificar asistentes de IA, pero mezclan responsabilidades en una única especificación monolítica. Este artículo presenta el Tutor Description Language (\tdl{}), una evolución de \adl{} que implementa la separación metodología-contenido mediante una arquitectura de cuatro capas desacopladas: (1) un motor de interpretación universal, (2) modelos instruccionales reutilizables alineados con teorías de Gagné y Bloom, (3) secuencias de aprendizaje específicas, y (4) contenido fuente independiente. A diferencia de sus predecesores, \tdl{} prioriza la simplicidad sintáctica (YAML) y la portabilidad entre plataformas (ChatGPT, Claude, Gemini) sobre la expresividad formal. Presentamos la especificación completa, dos modelos instruccionales de referencia, herramientas de validación, y una agenda de investigación para validar empíricamente si \tdl{} mejora los resultados de aprendizaje y reduce la carga docente.
\end{abstract}

\begin{IEEEkeywords}
Tutor Description Language, Assistant Description Language, Sistemas de Tutoría Inteligente, Lenguajes de Dominio Específico, Diseño Instruccional, Modelos de Lenguaje, IA en Educación
\end{IEEEkeywords}

% ============================================================================
% CUERPO DEL DOCUMENTO
% ============================================================================

% ============================================================================
% I. INTRODUCCIÓN
% ============================================================================
\section{Introducción}

\subsection{El Desafío de Escalar la Tutoría Personalizada}

El célebre ``problema 2 sigma'' identificado por Benjamin Bloom \cite{bloom1984} estableció que los estudiantes que reciben tutoría individualizada con aprendizaje por dominio (mastery learning) superan significativamente a aquellos que reciben instrucción convencional en grupo. Sin embargo, revisiones posteriores han matizado esta afirmación. VanLehn \cite{vanlehn2011}, en un meta-análisis de 54 estudios, encontró que el tamaño de efecto real de la tutoría humana es $d = 0.79$, no las dos desviaciones estándar originalmente propuestas. Kulik y Fletcher \cite{kulik2016}, analizando 50 evaluaciones controladas de Sistemas de Tutoría Inteligente (ITS), reportaron un tamaño de efecto mediano de $d = 0.66$. Ma et al. \cite{ma2014}, con 107 tamaños de efecto de 73 estudios, encontraron que los ITS no difieren significativamente de la tutoría humana individualizada ($g = -0.11$, no significativo).

Estos hallazgos tienen una implicación importante: los ITS bien diseñados ya han demostrado ser estadísticamente equivalentes a tutores humanos. El desafío contemporáneo no es alcanzar una meta idealizada de ``2 sigmas'', sino hacer accesible esta efectividad probada a escala, reduciendo el costo y complejidad de desarrollo que históricamente han limitado la adopción de ITS.

\subsection{LLMs como Tutores: Promesas y Limitaciones Documentadas}

Los Modelos de Lenguaje de Gran Escala (LLM) como GPT-4, Claude o Gemini han generado un interés creciente en su aplicación como tutores educativos \cite{lee2024impact}. Ofrecen ventajas aparentes: pueden mantener conversaciones en lenguaje natural, están disponibles continuamente, y pueden desplegarse sin la infraestructura técnica que requieren los ITS tradicionales. Kestin et al. \cite{kestin2025} reportaron resultados prometedores: en un experimento controlado, estudiantes que usaron un tutor de IA superaron significativamente a aquellos que recibieron instrucción activa presencial.

No obstante, la investigación reciente documenta una limitación fundamental: los LLM no están intrínsecamente alineados con objetivos pedagógicos. Tack y Piech \cite{tack2022} demostraron que los LLM actuales no son ``buenos tutores por defecto'': su objetivo de maximizar la utilidad entra en conflicto con estrategias tutoriales efectivas que implican desafiar productivamente al estudiante. Macina et al. \cite{macina2023} confirmaron que los modelos de lenguaje, sin modificaciones, proporcionan respuestas directas en lugar de guiar mediante preguntas. Borchers et al. \cite{borchers2025}, en un estudio reciente, encontraron que GPT-4 ``proporciona retroalimentación excesivamente directa que diverge de la tutoría efectiva'' y muestra ``adaptividad mínima'' a errores estudiantiles. Puech et al. \cite{puech2024} exploraron técnicas para orientar LLMs hacia comportamientos pedagógicos específicos como el Productive Failure, con resultados prometedores.

Esta limitación tiene raíces arquitectónicas. Los ITS tradicionales incorporan un \textit{student model} que rastrea el conocimiento, habilidades y misconceptions del aprendiz, permitiendo adaptación fina \cite{nwana1990}. Los LLM carecen de este componente: no mantienen un modelo persistente del estudiante entre sesiones y tienen capacidad limitada para diagnosticar el estado cognitivo del aprendiz en tiempo real \cite{scarlatos2025}.

\subsection{Antecedentes: La Separación Metodología-Contenido}

La idea de separar la metodología instruccional del contenido a enseñar no es nueva. Tiene raíces en la Component Display Theory (CDT) de Merrill \cite{merrill1983}, quien propuso que las estrategias instruccionales pueden especificarse independientemente del contenido de dominio. La Instructional Transaction Theory (ITT) \cite{merrill1991} formalizó este principio mediante los \textit{transaction shells}: ``algoritmos instruccionales que pueden usarse con diferentes tópicos de contenido siempre que estos tópicos sean de un tipo similar de conocimiento''.

IMS Learning Design \cite{koper2005} fue el intento más ambicioso de estandarizar esta separación. Propuso un metalenguaje basado en XML para especificar ``plantillas pedagógicas'' reutilizables. Sin embargo, a pesar de más de dos décadas desde su publicación, IMS LD no logró adopción generalizada. Derntl et al. \cite{derntl2012} investigaron las causas y encontraron un resultado sorprendente: en un estudio con 21 profesores universitarios, el 78\% logró conformidad con soluciones expertas tras solo 45 minutos de introducción. La complejidad conceptual \textit{no} fue la barrera principal. Las barreras documentadas fueron otras: ecosistema de herramientas inmaduro \cite{griffiths2005}, alto esfuerzo de desarrollo \cite{berggren2005}, y desajuste entre la terminología del lenguaje y los conceptos que los docentes usan para planificar \cite{neumann2008}.

En el ámbito de los ITS, el framework GIFT \cite{sottilare2012} implementa una arquitectura modular que separa el módulo pedagógico del módulo de dominio, permitiendo reutilizar estrategias instruccionales. REDEEM \cite{ainsworth2002} introdujo el concepto de \textit{pedagogical overlay}: los docentes importan contenido existente y superponen su expertise de enseñanza mediante configuraciones pedagógicas.

\subsection{Lenguajes de Prompts: PDL y POML}

Más recientemente, el campo del prompt engineering ha producido lenguajes formales para estructurar interacciones con LLM. IBM Research desarrolló el Prompt Declaration Language (PDL) \cite{ibm2024pdl}, un DSL basado en YAML con gramática formal y sistema de tipos. Microsoft Research propuso POML (Prompt Orchestration Markup Language) \cite{zhang2025poml}, un lenguaje de marcado inspirado en HTML con componentes semánticos como \texttt{<role>} y \texttt{<task>}. POML incorpora características avanzadas como un sistema de estilos similar a CSS para separar contenido de presentación, manejo nativo de datos multimodales, y un motor de plantillas con variables, bucles y condicionales.

Sin embargo, la sintaxis basada en XML de POML presenta una barrera de entrada para usuarios no técnicos: requiere comprender tags de apertura y cierre, atributos, anidamiento correcto, y caracteres de escape. Además, ni PDL ni POML incorporan conceptos de diseño instruccional: carecen de nociones como eventos de aprendizaje, secuencias pedagógicas, o separación explícita entre metodología y contenido educativo.

Estos lenguajes establecen un precedente relevante: es legítimo y útil crear lenguajes de dominio específico para estructurar interacciones con LLM. La pregunta es cómo hacerlo para el dominio educativo.

\subsection{Posicionamiento del Trabajo}

Este artículo presenta el Tutor Description Language (\tdl{}), una evolución de nuestro trabajo previo con \adl{} (Assistant Description Language) \cite{pernias2025adl}, especializada en tutorías educativas. \tdl{} no pretende ser conceptualmente novedoso: implementa el principio establecido de separación metodología-contenido, adaptándolo a la realidad de los LLM comerciales actuales.

\tdl{} se posiciona en la intersección de tres líneas de trabajo:

\begin{enumerate}
    \item \textbf{Herramientas de autoría de ITS}: Como CTAT \cite{aleven2009} y GIFT \cite{sottilare2012}, \tdl{} busca reducir la barrera técnica para crear tutores. A diferencia de estos, \tdl{} no requiere infraestructura propia: se despliega sobre plataformas LLM comerciales existentes.
    
    \item \textbf{Estándares de diseño instruccional}: Como IMS Learning Design \cite{koper2005}, \tdl{} formaliza la separación entre metodología y contenido. A diferencia de IMS LD, \tdl{} prioriza la simplicidad sintáctica (YAML vs. XML) y evita la completitud formal en favor de la usabilidad.
    
    \item \textbf{Lenguajes de prompts}: Como PDL \cite{ibm2024pdl} y POML \cite{zhang2025poml}, \tdl{} es un DSL para estructurar interacciones con LLM. A diferencia de estos, \tdl{} incorpora conceptos específicos de diseño instruccional y está orientado a educadores, no a desarrolladores.
\end{enumerate}

\subsection{Contribuciones y Estructura del Artículo}

Las contribuciones principales de este trabajo son:

\begin{itemize}
    \item Una \textbf{arquitectura de cuatro capas} que separa el cómo enseñar del qué enseñar, permitiendo la reutilización de modelos instruccionales.
    \item El concepto de \textbf{modelo instruccional como componente explícito} y reutilizable, alineado con teorías de diseño instruccional de Gagné y Bloom.
    \item Una \textbf{especificación formal completa} de \tdl{} como DSL, con sintaxis YAML y validación mediante JSON Schema.
    \item Dos \textbf{modelos instruccionales de referencia}: uno interactivo basado en la taxonomía de Bloom, y uno expositivo.
    \item \textbf{Demostración de portabilidad} entre plataformas LLM comerciales (ChatGPT, Claude, Gemini, OpenWebUI).
    \item Una \textbf{agenda de investigación} con hipótesis específicas para validar la efectividad de \tdl{}.
\end{itemize}

Es importante explicitar lo que \tdl{} \textit{no} es y \textit{no} pretende:

\begin{itemize}
    \item \tdl{} \textbf{no es un ITS completo}: carece de \textit{student model} para diagnóstico cognitivo individualizado.
    \item \tdl{} \textbf{no garantiza efectividad pedagógica}: estructura la interacción pero no asegura resultados de aprendizaje.
    \item \tdl{} \textbf{no ha sido validado empíricamente}: este artículo presenta la especificación y propone estudios futuros.
    \item \tdl{} \textbf{no resuelve las limitaciones de los LLM}: la adaptividad sigue dependiendo del modelo subyacente.
\end{itemize}

El resto del artículo se organiza como sigue: la Sección II revisa el estado del arte en ITS y herramientas de autoría; la Sección III analiza la evolución de \adl{} a \tdl{}; la Sección IV presenta la arquitectura de cuatro capas; la Sección V detalla los componentes de \tdl{}; la Sección VI profundiza en los modelos instruccionales; la Sección VII describe la portabilidad; la Sección VIII discute las implicaciones y limitaciones; y la Sección IX concluye con direcciones de trabajo futuro.

\subsection{Justificación Terminológica: ¿Por Qué ``Language''?}

El uso del término ``Language'' requiere justificación. Según van Deursen, Klint y Visser \cite{vandeursen2000}, un Domain-Specific Language (DSL) es ``un lenguaje de programación o especificación ejecutable que ofrece, mediante notaciones y abstracciones apropiadas, poder expresivo enfocado en un dominio particular''. Fowler \cite{fowler2010} identifica cuatro características: procesabilidad por software, fluidez notacional, expresividad deliberadamente limitada, y enfoque de dominio.

\tdl{} cumple estos criterios:

\begin{enumerate}
    \item \textbf{Sintaxis formal}: Gramática definida por YAML más restricciones de JSON Schema.
    \item \textbf{Semántica asignada}: Cada campo tiene significado específico interpretable.
    \item \textbf{Procesabilidad}: Parsers y validadores pueden verificar y ejecutar archivos \tdl{}.
    \item \textbf{Dominio específico}: Abstracciones para tutoría educativa (eventos instruccionales, secuencias de aprendizaje).
\end{enumerate}

La completitud de Turing no es requisito para denominar algo ``lenguaje''---HTML, CSS, SQL básico, y expresiones regulares son universalmente llamados lenguajes sin serlo. Los precedentes de PDL (``Prompt Declaration Language'') y POML (``Prompt Orchestration Markup Language'') validan este uso terminológico en el contexto de interacciones con LLM.

% ============================================================================
% II. ESTADO DEL ARTE EN ITS
% ============================================================================

\section{Sistemas de Tutoría Inteligente: Estado del Arte}

Esta sección revisa dos décadas de investigación en ITS (2005-2025), centrándose en la arquitectura, evidencia de efectividad, herramientas de autoría, y la transición hacia LLM.

\subsection{Arquitectura Clásica de ITS}

La arquitectura canónica de un ITS, establecida por Nwana \cite{nwana1990} y refinada en trabajos posteriores \cite{woolf2009}, comprende cuatro componentes interrelacionados:

\begin{enumerate}
    \item \textbf{Modelo de Dominio}: Representa el conocimiento experto que el sistema enseña.
    
    \item \textbf{Modelo del Estudiante}: Mantiene una representación del estado cognitivo del aprendiz: qué sabe, qué misconceptions tiene, y cómo progresa.
    
    \item \textbf{Modelo Pedagógico}: Contiene las estrategias instruccionales y decide qué acción tomar dado el estado del estudiante y el dominio.
    
    \item \textbf{Interfaz de Usuario}: Gestiona la comunicación entre el sistema y el estudiante.
\end{enumerate}

El \textit{student model} merece atención especial porque es el componente que diferencia un ITS de instrucción computarizada convencional. Técnicas como \textit{model tracing} \cite{anderson1995} comparan las acciones del estudiante contra un modelo cognitivo, identificando errores en tiempo real. \textit{Knowledge tracing} \cite{corbett1994} estima probabilísticamente el dominio de habilidades específicas.

\subsection{Evidencia de Efectividad}

La efectividad de los ITS está bien documentada mediante múltiples meta-análisis:

\textbf{VanLehn (2011)} \cite{vanlehn2011} comparó 54 estudios y encontró: tutoría humana $d = 0.79$; ITS \textit{step-based} $d = 0.76$; ITS \textit{substep-based} $d = 0.40$; CAI \textit{answer-based} $d = 0.31$. La comparación directa ITS vs. tutoría humana arrojó $g = -0.11$ (no significativo).

\textbf{Kulik y Fletcher (2016)} \cite{kulik2016} analizaron 50 evaluaciones controladas: tamaño de efecto mediano $d = 0.66$; 92\% de evaluaciones mostraron superioridad sobre instrucción convencional.

\textbf{Ma et al. (2014)} \cite{ma2014} identificaron el \textbf{diagnóstico cognitivo en tiempo real} y la \textbf{remediación adaptativa} como los elementos más críticos para la efectividad de los ITS.

Un hallazgo crucial de VanLehn \cite{vanlehn2011} fue que los tutores \textit{step-based} (que verifican comprensión en cada paso) son significativamente más efectivos que los \textit{answer-based} (que solo evalúan respuestas finales). Este resultado fundamenta el diseño del modelo Bloom 8-Step Interactive de \tdl{}.

\subsection{Herramientas de Autoría}

El desarrollo de ITS históricamente ha requerido entre 200-300 horas de desarrollo por hora de instrucción \cite{murray1999}. Las herramientas de autoría buscan reducir esta barrera.

\textbf{CTAT} \cite{aleven2009} introdujo los \textit{Example-Tracing Tutors}, que pueden construirse ``enteramente sin programación'' mediante programación por demostración. Los autores demuestran comportamientos deseados, y el sistema generaliza las reglas.

\textbf{GIFT} \cite{sottilare2012} implementa una arquitectura modular con separación explícita entre módulo pedagógico y módulo de dominio. Esta separación permite reutilizar estrategias instruccionales entre diferentes dominios de conocimiento.

\textbf{AutoTutor} \cite{graesser2004} pioneró el uso de diálogo en lenguaje natural para tutoría, con tamaños de efecto reportados de 0.4 a 1.5. Su arquitectura incluye un Curriculum Script que organiza las preguntas y un Dialog Advancer que gestiona la conversación.

Sin embargo, incluso estas herramientas requieren conocimientos técnicos significativos y ecosistemas de software específicos que limitan su adopción fuera de entornos de investigación.

\subsection{La Transición a LLMs}

La investigación documenta problemas sistemáticos con LLM como tutores. Tack y Piech \cite{tack2022} encontraron que ``los LLM actuales no son buenos tutores por defecto''. Borchers et al. \cite{borchers2025} confirmaron que GPT-4 ``proporciona retroalimentación excesivamente directa''.

Sin embargo, evidencia temprana es prometedora: Pardos y Bhandari \cite{pardos2024} encontraron que hints generados por ChatGPT produjeron 17\% de ganancia de aprendizaje vs. 11.62\% para hints de tutores humanos---sin diferencia significativa. Kestin et al. \cite{kestin2025} reportaron que estudiantes con tutor de IA superaron a aquellos con instrucción activa presencial.

\subsection{Lecciones de IMS Learning Design}

Educational Modeling Language (EML), desarrollado por Rob Koper en Open University of the Netherlands, fue la base de IMS Learning Design (IMS LD v1.0, febrero 2003) \cite{koper2005}. Van Es y Koper demostraron que 16 planes de lección de tradiciones pedagógicas diversas podían codificarse exitosamente en IMS LD.

Sin embargo, Derntl et al. \cite{derntl2012} reportaron: ``IMS LD ha estado disponible desde 2003, y sin embargo no ha sido ampliamente adoptado''. Las causas identificadas fueron:

\begin{itemize}
    \item \textbf{Ecosistema inmaduro}: Griffiths et al. \cite{griffiths2005} encontraron que ``el round-tripping entre herramientas no es posible''.
    \item \textbf{Alto esfuerzo}: Berggren et al. \cite{berggren2005} documentaron relación 3:1 entre preparación y uso.
    \item \textbf{Desajuste terminológico}: Neumann y Oberhuemer \cite{neumann2008} identificaron que ``los conceptos del lenguaje difieren de los que un maestro usa para planificar''.
\end{itemize}

Crucialmente, Derntl et al. \cite{derntl2012} encontraron que tras 45 minutos de introducción, 78\% de profesores logró conformidad con soluciones expertas. ``La estructura conceptual de IMS LD \textbf{no} impide su uso para autoría''. Las barreras fueron de ecosistema, no conceptuales.

\subsection{Lecciones para TDL}

La historia de IMS LD sugiere cuatro lecciones para el diseño de \tdl{}:

\begin{enumerate}
    \item La simplicidad conceptual no garantiza adopción---el ecosistema importa.
    \item El esfuerzo debe ser proporcional al beneficio percibido.
    \item La terminología debe alinearse con el lenguaje que usan los docentes.
    \item La portabilidad reduce dependencia del ecosistema específico.
\end{enumerate}

\tdl{} intenta mitigar estos riesgos: no requiere herramientas especiales (cualquier editor de texto sirve), usa terminología familiar (eventos, unidades, prompts), y funciona en múltiples plataformas LLM comerciales. Sin embargo, hereda el riesgo fundamental de que el esfuerzo de formalización no se perciba proporcional al beneficio.

% ============================================================================
% III. DE ADL A TDL
% ============================================================================

\section{De ADL a TDL: Motivación y Principios de Diseño}

Esta sección presenta el trabajo previo con \adl{} y analiza las motivaciones para su evolución hacia \tdl{}.

\subsection{Assistant Description Language (ADL)}

Nuestro trabajo previo con \adl{} \cite{pernias2025adl} introdujo un lenguaje estructurado basado en \yamlformat{} para que los docentes codificaran asistentes educativos basados en LLM. \adl{} demostró que es posible capturar la expertise pedagógica de un docente en una especificación formal que un LLM puede ejecutar fielmente.

\adl{} definió cuatro tipos de herramientas pedagógicas:

\begin{itemize}
    \item \textbf{Commands} (/command): Acciones pedagógicas autocontenidas, desde explicaciones simples hasta procedimientos multi-paso. Cada command encapsula una estrategia de enseñanza específica.
    \item \textbf{Options} (/option): Modificadores que ajustan el comportamiento global sin asociarse a contenido específico, como nivel de detalle o estilo de comunicación.
    \item \textbf{Decorators} (+++decorator): Estilos pedagógicos que modifican cómo se ejecuta un command. Por ejemplo, +++socratic transforma cualquier command en una serie de preguntas guiadas.
    \item \textbf{Workflows} (/workflow): Secuencias automatizadas de commands para lecciones completas, permitiendo estructurar sesiones de aprendizaje coherentes.
\end{itemize}

Un piloto con 30 estudiantes en un curso de Turismo mostró resultados positivos: el 70\% usó el asistente frecuentemente, el 83\% valoró las respuestas como útiles, y el 100\% recomendó continuar usándolo. Los docentes percibieron la herramienta como una extensión de su práctica pedagógica, no como un reemplazo.

\subsection{Análisis Crítico de ADL para Tutorías}

La experiencia con \adl{} reveló tanto fortalezas como limitaciones cuando se aplica específicamente a tutorías educativas estructuradas.

\subsubsection{Fortalezas de ADL}

Los commands permitieron encapsular estrategias de enseñanza complejas, los decorators añadieron flexibilidad estilística, y los workflows automatizaron secuencias completas. La separación entre la especificación (creada por el docente) y la ejecución (realizada por el LLM) preservó la autoría pedagógica.

\subsubsection{Limitaciones Identificadas}

Sin embargo, identificamos cuatro limitaciones principales para el uso educativo especializado:

\textbf{Arquitectura monolítica}: Todo ---identidad del asistente, comportamientos, herramientas, contenido--- se define en un único archivo \adl{}. Esto dificulta la reutilización: si dos cursos diferentes quieren usar la misma metodología socrática, deben duplicar las definiciones de decorators y workflows.

\textbf{Método pedagógico implícito}: El enfoque pedagógico queda disperso entre múltiples commands y decorators. No existe una representación explícita de la secuencia instruccional como unidad coherente. Un observador externo tendría dificultades para identificar qué modelo pedagógico sigue el asistente.

\textbf{Contenido embebido}: El contenido a enseñar se incorpora directamente en los prompts de los commands. Actualizar el contenido requiere modificar la especificación \adl{}, aumentando el riesgo de introducir errores y dificultando que expertos en contenido (sin conocimiento de \adl{}) contribuyan directamente.

\textbf{Dificultad para escalar metodologías}: Un diseñador instruccional que desarrolle una metodología efectiva no puede compartirla fácilmente con otros docentes. Cada nuevo curso requiere recrear la estructura metodológica desde cero.

\subsection{Lecciones del Piloto ADL}

Durante el piloto de \adl{}, observamos un fenómeno revelador: los docentes invocaban los commands en secuencias notablemente consistentes. Por ejemplo, una profesora de Turismo siempre comenzaba con /DAFO\_strengths +++step-by-step, seguido de /DAFO\_weaknesses +++socratic, y finalizaba con /DAFO\_review +++critique. Esta secuencia se repetía idénticamente en diferentes sesiones y grupos.

Esta observación sugirió que los docentes tienen \textbf{firmas metodológicas} ---patrones estables de enseñanza que aplican independientemente del contenido específico. Los workflows de \adl{} capturaban estas secuencias, pero las trataban como específicas de cada curso en lugar de como metodologías reutilizables.

La conclusión fue clara: necesitábamos separar el \textbf{cómo enseñar} (la metodología, aplicable a múltiples contenidos) del \textbf{qué enseñar} (el contenido específico de cada curso). Esta separación es precisamente lo que \tdl{} implementa.

\subsection{Principios de Diseño de TDL}

Basándonos en este análisis, establecimos cinco principios fundamentales para el diseño de \tdl{}:

\begin{table}[htbp]
\caption{Principios de Diseño de TDL}
\label{tab:principios}
\centering
\begin{tabular}{p{2cm}p{5.5cm}}
\toprule
\textbf{Principio} & \textbf{Descripción} \\
\midrule
Separación de responsabilidades & Cada capa de la arquitectura tiene una única responsabilidad bien definida \\
\addlinespace
Reutilización metodológica & Los modelos instruccionales son plantillas reutilizables entre múltiples cursos \\
\addlinespace
Alineamiento con DI & Los eventos instruccionales se alinean con teorías de diseño instruccional establecidas \\
\addlinespace
Independencia del contenido & El contenido fuente se mantiene separado de la estructura metodológica \\
\addlinespace
Portabilidad & La especificación funciona en cualquier plataforma LLM sin modificaciones \\
\bottomrule
\end{tabular}
\end{table}

El principio de \textbf{separación de responsabilidades} implica que modificar el contenido no requiere tocar la metodología, y viceversa. La \textbf{reutilización metodológica} permite que un modelo instruccional validado se aplique a cursos de diferentes disciplinas. El \textbf{alineamiento con diseño instruccional} garantiza que \tdl{} hable el lenguaje de los profesionales de la educación. La \textbf{independencia del contenido} facilita que expertos en la materia contribuyan sin conocer \tdl{}. La \textbf{portabilidad} evita el vendor lock-in y maximiza la adopción.

\subsection{De Monolítico a Desacoplado}

La transición de \adl{} a \tdl{} puede resumirse como un cambio de arquitectura monolítica a arquitectura desacoplada:

\begin{table}[htbp]
\caption{Comparativa entre ADL y TDL}
\label{tab:comparativa}
\centering
\begin{tabular}{p{1.8cm}p{2.5cm}p{2.5cm}}
\toprule
\textbf{Aspecto} & \textbf{ADL} & \textbf{TDL} \\
\midrule
Alcance & Asistentes genéricos & Tutorías educativas \\
\addlinespace
Arquitectura & Monolítica (1 archivo) & 4 capas desacopladas \\
\addlinespace
Método pedagógico & Implícito en commands & Explícito en modelo \\
\addlinespace
Reutilización & Por command individual & Por modelo completo \\
\addlinespace
Contenido & Embebido en prompts & Desacoplado \\
\addlinespace
Alineamiento con DI & Parcial & Directo (Gagné, Bloom) \\
\bottomrule
\end{tabular}
\end{table}

\adl{} permanece como una opción válida para asistentes no estrictamente educativos o para casos donde la simplicidad de un único archivo es preferible. \tdl{} está optimizado específicamente para tutorías donde el diseño instruccional formal aporta valor.

% ============================================================================
% IV. ARQUITECTURA TDL
% ============================================================================

\section{Arquitectura TDL}

\tdl{} implementa una arquitectura de cuatro capas desacopladas, donde cada capa tiene una responsabilidad específica y puede evolucionar independientemente. La Fig.~\ref{fig:arquitectura} ilustra esta organización.

\begin{figure}[htbp]
\centering
\begin{tikzpicture}[
    layer/.style={
        rectangle,
        draw=black,
        thick,
        minimum width=6.5cm,
        minimum height=1cm,
        align=center,
        font=\small
    }
]

\node[layer, fill=blue!10] (engine) at (0,3.6) {\textbf{CAPA 1: ENGINE}\\Motor de Interpretación};

\node[layer, fill=blue!20] (model) at (0,2.4) {\textbf{CAPA 2: INSTRUCTIONAL MODEL}\\Modelo Instruccional};

\node[layer, fill=blue!30] (sequence) at (0,1.2) {\textbf{CAPA 3: LEARNING SEQUENCE}\\Secuencia de Aprendizaje};

\node[layer, fill=blue!40] (content) at (0,0) {\textbf{CAPA 4: CONTENT SOURCE}\\Contenido Fuente};

\draw[-stealth, thick] (engine) -- (model);
\draw[-stealth, thick] (model) -- (sequence);
\draw[-stealth, thick] (sequence) -- (content);

\end{tikzpicture}
\caption{Arquitectura de cuatro capas de TDL.}
\label{fig:arquitectura}
\end{figure}

\subsection{Relación con la Arquitectura Clásica de ITS}

La arquitectura de \tdl{} puede mapearse a la arquitectura clásica de ITS, aunque con diferencias importantes:

\begin{table}[h]
\centering
\caption{Correspondencia entre Arquitectura ITS y TDL}
\label{tab:its_tdl_mapping}
\begin{tabular}{@{}ll@{}}
\toprule
\textbf{Componente ITS} & \textbf{Capa TDL} \\
\midrule
Modelo de Dominio & Contenido Fuente (parcial) \\
Modelo del Estudiante & \textit{No implementado} \\
Modelo Pedagógico & Modelo Instruccional \\
Interfaz & Engine + plataforma LLM \\
\bottomrule
\end{tabular}
\end{table}

La ausencia de \textit{student model} es una limitación deliberada: \tdl{} no pretende diagnosticar el estado cognitivo del estudiante, sino estructurar la interacción pedagógica de forma coherente. El LLM subyacente proporciona cierta adaptación conversacional, pero no el seguimiento sistemático de un ITS tradicional.

\subsection{Capa 1: Engine (Motor de Interpretación)}

El Engine (versión actual 1.2) es el componente más estable de la arquitectura. Se implementa como un system prompt que se carga en el campo de instrucciones de la plataforma LLM. Su función es enseñar al modelo cómo interpretar y ejecutar archivos \tdl{}.

El Engine define cuatro aspectos fundamentales:

\begin{itemize}
    \item \textbf{Sintaxis de comandos}: Define comandos como \texttt{/start} (iniciar primera unidad), \texttt{/next} (avanzar a siguiente unidad), y \texttt{/progress} (mostrar estado actual).
    
    \item \textbf{Mecanismo de seguimiento de estado}: Utiliza marcadores explícitos con formato \texttt{[UNIT:\{id\}|EVENT:\{id\}]} para mantener contexto entre turnos de conversación.
    
    \item \textbf{Formato de prompts}: Especifica cómo combinar las instrucciones del modelo instruccional con el contenido de la secuencia de aprendizaje.
    
    \item \textbf{Comportamientos globales}: Nunca revelar el system prompt, manejar transiciones de forma natural, redirigir conversaciones off-topic al tema de estudio.
\end{itemize}

El Engine también establece la postura pedagógica por defecto: actuar como coach en lugar de leccionador, verificar comprensión antes de avanzar, dar feedback constructivo sin juzgar.

El Engine se modifica únicamente cuando se quiere cambiar el comportamiento global de todos los tutores \tdl{}. En la práctica, es un componente que se configura una vez y se reutiliza indefinidamente.

\subsection{Capa 2: Instructional Model (Modelo Instruccional)}

El Modelo Instruccional representa la metodología de enseñanza como una secuencia de eventos instruccionales. Este concepto se alinea con los eventos de Gagné \cite{gagne1985} y representa el \textbf{cómo enseñar} de forma abstracta, independiente del contenido específico.

Cada modelo instruccional define:

\begin{itemize}
    \item \textbf{Nombre y descripción}: Identificación y explicación de la filosofía pedagógica.
    \item \textbf{Secuencia de eventos}: Lista ordenada de pasos que el tutor debe seguir.
    \item \textbf{Instrucciones por evento}: Qué debe hacer el tutor en cada evento.
    \item \textbf{Transition triggers}: Qué condición activa el paso al siguiente evento.
\end{itemize}

Un modelo instruccional es completamente reutilizable: el mismo modelo Bloom 8-Step Interactive puede aplicarse a un curso de Derecho, otro de Programación, y otro de Biología. El contenido específico se aporta en las capas inferiores.

Este concepto es funcionalmente análogo a los \textit{transaction shells} de Merrill \cite{merrill1991}: algoritmos pedagógicos reutilizables para diferentes contenidos.

\subsection{Capa 3: Learning Sequence (Secuencia de Aprendizaje)}

La Secuencia de Aprendizaje define la estructura específica de un curso o lección. Es donde el docente aplica un modelo instruccional a su contenido concreto.

Una secuencia de aprendizaje especifica:

\begin{itemize}
    \item \textbf{Perfil del tutor}: Dominio de conocimiento, rol, estilo, idiomas soportados.
    \item \textbf{Herencia del modelo}: Qué modelo instruccional utilizar (mediante \texttt{extends}).
    \item \textbf{Comportamientos}: Saludo inicial, respuesta a ayuda, manejo de off-topic, disclaimers.
    \item \textbf{Herramientas}: Comandos disponibles para el estudiante (/start, /progress).
    \item \textbf{Unidades de aprendizaje}: Estructura del curso con objetivos, prompts específicos por evento, y navegación entre unidades.
\end{itemize}

La secuencia hereda la metodología del modelo instruccional pero proporciona el contenido específico. Esta separación es la clave de la reutilización.

\subsection{Capa 4: Content Source (Contenido Fuente)}

El Contenido Fuente es el material del experto en la materia: apuntes, textos, referencias. Esta capa es opcional (los prompts de la secuencia pueden contener el contenido directamente), pero resulta especialmente útil para:

\begin{itemize}
    \item Contenido extenso que no cabe cómodamente en prompts.
    \item Material que se actualiza frecuentemente (ej., normativa legal).
    \item Situaciones donde el experto en contenido y el diseñador instruccional son personas diferentes.
\end{itemize}

\tdl{} soporta múltiples formatos de contenido fuente: Markdown (recomendado), texto plano, PDF, Word. La secuencia de aprendizaje referencia secciones específicas del contenido mediante el campo \texttt{source\_section}.

\subsection{Flujo de Datos e Interacción}

La Fig.~\ref{fig:flujo} muestra cómo interactúan las capas cuando un estudiante interactúa con un tutor \tdl{}.

\begin{figure}[htbp]
\centering
\begin{tikzpicture}[
    node distance=1cm,
    box/.style={
        rectangle,
        draw=black,
        thick,
        rounded corners,
        minimum width=2cm,
        minimum height=0.7cm,
        align=center,
        font=\footnotesize
    }
]

\node[box, fill=gray!20] (user) {Estudiante};
\node[box, fill=blue!10, below=of user] (engine) {Engine};
\node[box, fill=blue!20, below=of engine] (sequence) {Learning Sequence};
\node[box, fill=blue!30, below left=0.6cm and 0.3cm of sequence] (model) {Model};
\node[box, fill=blue!40, below right=0.6cm and 0.3cm of sequence] (content) {Content};
\node[box, fill=green!20, below=1.8cm of sequence] (response) {Respuesta};

\draw[-stealth, thick] (user) -- (engine);
\draw[-stealth, thick] (engine) -- (sequence);
\draw[-stealth, thick] (sequence) -- (model);
\draw[-stealth, thick] (sequence) -- (content);
\draw[-stealth, thick] (model) |- (response);
\draw[-stealth, thick] (content) |- (response);
\draw[-stealth, thick] (response.east) -- ++(1.5,0) |- (user.east);

\end{tikzpicture}
\caption{Flujo de datos en TDL.}
\label{fig:flujo}
\end{figure}

El proceso sigue estos pasos:

\begin{enumerate}
    \item El estudiante envía un mensaje (ej., ``Hola'' o \texttt{/start}).
    \item El Engine interpreta el mensaje y determina el contexto actual (unidad, evento).
    \item El Engine localiza la Learning Sequence en el knowledge de la plataforma.
    \item La Sequence indica qué Instructional Model usar mediante \texttt{extends}.
    \item El Engine carga los eventos del modelo (E1, E2, ...).
    \item Para el evento actual, el Engine usa las instrucciones del modelo combinadas con el prompt de la unidad.
    \item Si hay Content Source referenciado, el Engine incorpora el material relevante.
    \item El LLM genera la respuesta siguiendo las instrucciones compuestas.
    \item El Engine actualiza el estado \texttt{[UNIT:id|EVENT:id]} si hay transición.
\end{enumerate}

\subsection{Principio de Separación}

La arquitectura implementa el principio de separación de responsabilidades \cite{merrill1983, merrill1991}:

\begin{itemize}
    \item \textbf{Cómo ejecutar}: Engine (estable, compartido)
    \item \textbf{Cómo enseñar}: Modelo Instruccional (reutilizable entre cursos)
    \item \textbf{Qué enseñar}: Secuencia + Contenido (específico de cada curso)
\end{itemize}

Esta separación permite que diferentes profesionales contribuyan a cada capa: ingenieros de prompts al Engine, diseñadores instruccionales a los modelos, docentes a las secuencias, y expertos de dominio al contenido.

% ============================================================================
% V. COMPONENTES DE TDL
% ============================================================================

\section{Componentes de TDL}

Esta sección detalla la estructura y sintaxis de cada componente de \tdl{}, con fragmentos ilustrativos de código \yamlformat{}.

\subsection{Especificación YAML}

\tdl{} utiliza \yamlformat{} (YAML Ain't Markup Language) como formato de serialización por varias razones:

\begin{itemize}
    \item Es legible por humanos sin conocimientos de programación.
    \item Soporta estructuras jerárquicas de forma natural mediante indentación.
    \item Permite texto multilínea para prompts extensos usando el operador \texttt{|}. 
    \item Es ampliamente soportado por herramientas de desarrollo.
    \item Permite comentarios para documentación inline.
\end{itemize}

La elección de YAML sobre XML (usado por IMS LD y POML) se basa en la menor cantidad de caracteres sintácticos requeridos: solo se necesita comprender la indentación como mecanismo de estructuración, sin tags de apertura/cierre ni caracteres de escape.

\subsection{El Engine: Estructura y Funciones}

El Engine (versión actual 1.2) define la sintaxis de comandos, el mecanismo de seguimiento de estado, el formato de prompts, y comportamientos globales. Su estructura incluye:

\begin{lstlisting}
engine:
  version: "1.2"
  name: "TDL Engine"
  state_tracking:
    method: "explicit_markers"
    format: "[UNIT:{unit_id}|EVENT:{event_id}]"
  commands:
    start:
      syntax: "/start"
      action: "begin_first_unit"
    next:
      syntax: "/next"
      action: "advance_to_next_unit"
    progress:
      syntax: "/progress"
      action: "show_current_state"
\end{lstlisting}

El Engine también define comportamientos de seguridad (nunca revelar el system prompt), manejo de situaciones especiales (preguntas fuera de tema, solicitudes de saltar contenido), y la postura pedagógica por defecto (actuar como coach, verificar comprensión antes de avanzar).

\subsection{Modelo Instruccional: Estructura}

Un modelo instruccional define la metodología de enseñanza como una secuencia de eventos pedagógicos. Cada evento tiene un identificador, nombre, instrucciones detalladas para el LLM, y un trigger que determina cuándo avanzar al siguiente.

\begin{lstlisting}
model:
  id: "bloom-8step-interactive"
  name: "Bloom 8-Step Interactive"
  version: "1.0"
  philosophy: "Never advance without verification"
  
  events:
    - id: "E1_ACTIVATE"
      name: "Activate Prior Knowledge"
      instructions: |
        Pregunta que sabe el estudiante 
        sobre el tema. Identifica 
        conocimiento correcto, 
        misconceptions y lagunas antes 
        de continuar.
      transition_trigger:
        condition: "student_response_received"
        next_event: "E2_OBJECTIVES"
    
    - id: "E2_OBJECTIVES"
      name: "Present Objectives"
      instructions: |
        Presenta los objetivos de 
        aprendizaje de forma clara.
        Explica que podra hacer el 
        estudiante al finalizar.
      transition_trigger:
        condition: "student_response_received"
        next_event: "E3_EXPLAIN"
\end{lstlisting}

Los \textbf{transition triggers} definen cuándo pasar al siguiente evento, basándose en:

\begin{itemize}
    \item \texttt{student\_response\_received}: Espera cualquier respuesta del estudiante.
    \item \texttt{comprehension\_verified}: Espera verificación de comprensión.
    \item \texttt{explicit\_command}: Espera comando explícito (/next).
    \item \texttt{auto}: Continúa automáticamente sin esperar input.
\end{itemize}

\subsection{Secuencia de Aprendizaje: Estructura}

La secuencia de aprendizaje es el archivo que el docente crea para su curso específico. Hereda de un modelo instruccional mediante \texttt{extends} y define el perfil del tutor, comportamientos, y unidades de aprendizaje.

\begin{lstlisting}
sequence:
  id: "ia-generativa-fundamentals"
  name: "Fundamentos de IA Generativa"
  version: "1.0"
  description: "Tutor de conceptos basicos 
                de IA Generativa"
  author: "Pedro Pernias"
  
  tutor_profile:
    name: "Alex"
    personality: |
      Tutor entusiasta y paciente de IA.
      Usa analogias cotidianas para 
      explicar conceptos tecnicos.
      El ejemplo de Felix es tu favorito
      para explicar atencion.
      Nunca digas que la IA "entiende" 
      como los humanos.
  
  extends: "instructional_model_bloom.yaml"
  
  source_content:
    - file: "content_ia_generativa.md"
      type: "reference"
  
  behaviors:
    response_length: "medium"
    language: "es"
\end{lstlisting}

Las \textbf{unidades de aprendizaje} (\texttt{learning\_units}) estructuran el contenido del curso. Cada unidad es un tema discreto que pasa por todos los eventos del modelo pedagógico:

\begin{lstlisting}
  learning_units:
    - id: "LU1"
      title: "Foundation Models"
      objectives:
        - level: "understand"
          description: "Explicar que es 
                        un foundation model"
      prompt: |
        Puntos clave:
        - Modelos grandes preentrenados
        - Sirven como BASE para muchas tareas
        
        Analogia: cimientos de un edificio.
        
        Ejemplos: GPT, Claude, LLaMA.
      next: "LU2"
    
    - id: "LU4"
      title: "Mecanismo de Atencion"
      objectives:
        - level: "understand"
          description: "Explicar como funciona
                        la atencion"
      prompt: |
        Usa el ejemplo de Felix:
        "Felix vio un gato negro y un gato 
        blanco. Le dio comida a el."
        
        Problema: A que se refiere "el"?
        Solucion: Atencion permite "mirar 
        atras" y resolver referencias.
      source_sections:
        - "Seccion: Arquitectura Transformer"
      next: "LU5"
\end{lstlisting}

Nótese que el campo \texttt{prompt} NO es lo que el tutor dice al estudiante, sino una instrucción para el tutor sobre qué enseñar. Debe incluir: puntos clave estructurados, ejemplos concretos, analogías sugeridas, y misconceptions a abordar.

\subsection{Mecanismo de Herencia (extends)}

El campo \texttt{extends} permite que una secuencia de aprendizaje herede la metodología de un modelo instruccional. Cuando el Engine procesa una secuencia con \texttt{extends}:

\begin{enumerate}
    \item Localiza el archivo del modelo instruccional referenciado.
    \item Carga la secuencia de eventos del modelo (E1, E2, ..., En).
    \item Para cada unidad de aprendizaje, ejecuta todos los eventos en orden.
    \item Usa las instrucciones del modelo combinadas con el prompt de la unidad.
    \item Gestiona las transiciones según los triggers definidos.
\end{enumerate}

Este mecanismo permite que el mismo modelo Bloom 8-Step Interactive se aplique a cursos completamente diferentes: el modelo aporta el \textbf{cómo} (activar conocimiento previo, explicar, verificar, practicar) y la secuencia aporta el \textbf{qué} (Foundation Models, Transformers, Atención...).

\subsection{Contenido Fuente: Estructura}

El contenido fuente es un archivo separado (típicamente Markdown) que contiene el material de referencia. Se recomienda usar cuando:

\begin{itemize}
    \item El contenido es extenso y no cabe cómodamente en prompts.
    \item El material se actualiza frecuentemente (ej., normativa legal).
    \item El experto en contenido y el diseñador instruccional son personas diferentes.
    \item Se desea separar claramente el ``qué'' del ``cómo''.
\end{itemize}

El formato recomendado usa Markdown con secciones delimitadas por encabezados:

\begin{lstlisting}
## Seccion: Foundation Models

Los Foundation Models son modelos de gran 
escala preentrenados con enormes cantidades 
de datos. Sirven como base para multiples 
tareas especificas...

## Seccion: Arquitectura Transformer

El Transformer revoluciono el procesamiento 
del lenguaje natural al introducir el 
mecanismo de atencion...
\end{lstlisting}

La secuencia referencia secciones específicas mediante el campo \texttt{source\_sections}:

\begin{lstlisting}
learning_units:
  - id: "LU2"
    title: "Arquitectura Transformer"
    source_sections:
      - "Seccion: Arquitectura Transformer"
    prompt: |
      Explica la arquitectura Transformer.
      Enfatiza que reemplazo a las RNNs.
\end{lstlisting}

El Engine localiza la sección relevante y la incorpora al contexto cuando el prompt lo requiere. Esto permite actualizar el contenido sin modificar la estructura de la secuencia.

\subsection{JSON Schemas para Validación}

Para garantizar la correctitud de los archivos \tdl{}, hemos desarrollado esquemas JSON Schema que definen la estructura válida de cada tipo de archivo. Estos esquemas permiten validación automática antes del despliegue.

El esquema para modelos instruccionales verifica:

\begin{itemize}
    \item Presencia de campos obligatorios (\texttt{model.id}, \texttt{model.events}).
    \item Formato correcto del identificador (minúsculas, guiones).
    \item Al menos dos eventos en la secuencia.
    \item Cada evento tiene \texttt{id}, \texttt{instructions}, y \texttt{transition\_trigger}.
\end{itemize}

El esquema para secuencias de aprendizaje verifica adicionalmente:

\begin{itemize}
    \item Presencia de \texttt{extends} referenciando un modelo válido.
    \item Estructura correcta de \texttt{tutor\_profile} con \texttt{name} y \texttt{personality}.
    \item Al menos una unidad de aprendizaje definida con \texttt{id}, \texttt{title}, y \texttt{prompt}.
    \item Consistencia en referencias \texttt{next} entre unidades.
\end{itemize}

\subsection{Herramienta de Validación}

Hemos desarrollado un validador en Python que utiliza los esquemas JSON Schema para verificar archivos \tdl{}:

\begin{lstlisting}
pip install tdl-validator
tdl-validate learning_sequence.yaml \
    --model modelo.yaml
\end{lstlisting}

El validador detecta errores comunes como campos faltantes, tipos incorrectos, indentación incorrecta, o referencias a modelos inexistentes, proporcionando mensajes descriptivos que facilitan la corrección.

\subsection{Patrones de Diseño Comunes}

El manual de \tdl{} identifica cinco patrones de diseño frecuentes:

\begin{enumerate}
    \item \textbf{Conceptos Técnicos}: Modelo Bloom 8-Step, verificación frecuente, analogías cotidianas, unidades pequeñas y focalizadas.
    
    \item \textbf{Glosario}: Modelo simplificado (Identify, Explain, Connect), sin secuencia lineal, respuestas cortas, conexiones entre términos.
    
    \item \textbf{Práctica con Ejercicios}: Generación dinámica de problemas, retroalimentación específica por tipo de error, dificultad progresiva.
    
    \item \textbf{Casos Prácticos}: Escenarios contextualizados, toma de decisiones guiada, múltiples caminos posibles.
    
    \item \textbf{Preparación de Exámenes}: Preguntas tipo examen real, explicación detallada de respuestas, identificación de áreas débiles.
\end{enumerate}

% ============================================================================
% VI. MODELOS INSTRUCCIONALES
% ============================================================================

\section{Modelos Instruccionales en TDL}

El concepto de modelo instruccional es central en \tdl{}. Esta sección profundiza en su fundamentación teórica y presenta los dos modelos de referencia incluidos en la especificación.

\subsection{El Concepto de Evento Instruccional}

Los eventos instruccionales en \tdl{} se inspiran en la teoría de Robert Gagné \cite{gagne1985}, quien identificó nueve eventos que optimizan el aprendizaje. \tdl{} adapta este marco al contexto de tutorías conversacionales, donde cada evento representa una fase con propósito pedagógico específico.

La Tabla~\ref{tab:gagne_mapping} muestra la correspondencia entre los eventos de Gagné y los eventos típicos de \tdl{}.

\begin{table}[htbp]
\caption{Correspondencia entre Eventos de Gagné y TDL}
\label{tab:gagne_mapping}
\centering
\begin{tabular}{p{3.2cm}p{3.5cm}}
\toprule
\textbf{Evento de Gagné} & \textbf{Evento TDL} \\
\midrule
Ganar atención & Attention / Activate \\
Informar objetivos & Objectives \\
Estimular recuerdo & Recall / Activate \\
Presentar contenido & Explain / Present \\
Proporcionar guía & Guidance / Elaborate \\
Elicitar desempeño & Practice / Elicit \\
Dar retroalimentación & Verify / Feedback \\
Evaluar desempeño & Assess \\
Mejorar retención & Summary / Transfer \\
\bottomrule
\end{tabular}
\end{table}

Esta correspondencia no es rígida: \tdl{} permite flexibilidad en cómo se organizan los eventos, siempre que mantengan coherencia pedagógica.

\subsection{Relación con Transaction Shells}

Los modelos instruccionales de \tdl{} son funcionalmente análogos a los \textit{transaction shells} de Merrill \cite{merrill1991}: algoritmos pedagógicos reutilizables para diferentes contenidos. Un transaction shell especifica la secuencia de interacciones sin conocer el contenido específico; el modelo instruccional de \tdl{} hace exactamente lo mismo.

Esta analogía refuerza que \tdl{} no introduce conceptos nuevos, sino que implementa principios establecidos en un contexto tecnológico nuevo (LLMs).

\subsection{Modelo de Referencia: Bloom 8-Step Interactive}

El modelo Bloom 8-Step Interactive implementa un enfoque socrático donde el tutor verifica constantemente la comprensión antes de avanzar. Su filosofía central es: \textbf{``Never advance without verification''} (nunca avanzar sin verificar).

Este diseño se alinea con la evidencia de VanLehn \cite{vanlehn2011}: los tutores \textit{step-based} ($d = 0.76$) superan significativamente a los \textit{answer-based} ($d = 0.31$).

A pesar del nombre ``8-Step'', el modelo implementa 10 eventos para mayor granularidad en la verificación:

\begin{table}[htbp]
\caption{Eventos del Modelo Bloom 8-Step Interactive}
\label{tab:bloom_events}
\centering
\begin{tabular}{clp{2.8cm}}
\toprule
\textbf{ID} & \textbf{Evento} & \textbf{Propósito} \\
\midrule
E1 & ACTIVATE & Activar conocimiento previo \\
E2 & OBJECTIVES & Presentar objetivos \\
E3 & EXPLAIN & Presentar contenido \\
E4 & VERIFY\_EXPLAIN & Verificar comprensión inicial \\
E5 & ELABORATE & Profundizar si correcto \\
E6 & VERIFY\_ELABORATE & Verificar profundización \\
E7 & PRACTICE & Práctica guiada \\
E8 & VERIFY\_PRACTICE & Verificar práctica \\
E9 & SUMMARY & Resumen y conexiones \\
E10 & CLOSE & Cierre y transición \\
\bottomrule
\end{tabular}
\end{table}

\subsubsection{Características del Modelo}

\begin{itemize}
    \item \textbf{Verificación continua}: Cada explicación (E3, E5) va seguida de verificación (E4, E6).
    \item \textbf{Feedback inmediato}: El tutor evalúa cada respuesta antes de continuar.
    \item \textbf{Progresión cognitiva}: De recuerdo a comprensión, de comprensión a aplicación, de aplicación a síntesis (alineado con taxonomía de Bloom \cite{anderson2001}).
    \item \textbf{Detección de misconceptions}: E1 identifica conocimiento incorrecto antes de enseñar.
\end{itemize}

\subsubsection{Casos de Uso Ideales}

Este modelo es ideal para:

\begin{itemize}
    \item Conceptos técnicos complejos donde cada concepto es prerequisito del siguiente.
    \item Dominios donde los errores conceptuales son costosos (medicina, ingeniería, derecho).
    \item Estudiantes que necesitan guía activa y verificación frecuente.
    \item Situaciones donde la comprensión profunda es más importante que la cobertura amplia.
\end{itemize}

\subsection{Modelo de Referencia: Expository}

El modelo Expository implementa un enfoque de clase magistral estructurada, basado directamente en los nueve eventos de Gagné. El tutor expone el contenido de forma organizada antes de solicitar participación activa.

Su filosofía es: \textbf{``Transmisión estructurada''} ---presentación organizada con verificaciones concentradas al final.

\begin{table}[htbp]
\caption{Eventos del Modelo Expository (basado en Gagné)}
\label{tab:expository_events}
\centering
\begin{tabular}{clp{3cm}}
\toprule
\textbf{ID} & \textbf{Evento} & \textbf{Propósito} \\
\midrule
E1 & ATTENTION & Captar atención inicial \\
E2 & OBJECTIVES & Informar objetivos \\
E3 & RECALL & Estimular recuerdo previo \\
E4 & PRESENT & Presentar contenido \\
E5 & GUIDANCE & Proporcionar guía \\
E6 & ELICIT & Elicitar desempeño \\
E7 & FEEDBACK & Dar retroalimentación \\
E8 & ASSESS & Evaluar desempeño \\
E9 & TRANSFER & Mejorar retención \\
\bottomrule
\end{tabular}
\end{table}

\subsubsection{Diferencias con el Modelo Interactivo}

La diferencia clave es que el tutor puede dar exposiciones más largas (5-8 párrafos) antes de pausar para interacción. El estudiante tiene un rol más receptivo inicialmente, con participación activa concentrada en E6-E8.

\subsubsection{Casos de Uso Ideales}

Este modelo es ideal para:

\begin{itemize}
    \item Contenido normativo, legal o regulatorio donde la precisión es crítica.
    \item Primera exposición a un tema completamente nuevo.
    \item Cobertura rápida de material introductorio o contextual.
    \item Estudiantes que prefieren recibir información completa antes de interactuar.
\end{itemize}

\subsection{Creación de Modelos Personalizados}

\tdl{} no limita a los docentes a los modelos de referencia. Un diseñador instruccional puede crear modelos personalizados siguiendo este proceso:

\begin{enumerate}
    \item Identificar la filosofía pedagógica guía.
    \item Dibujar la secuencia de eventos en papel.
    \item Definir qué ocurre en cada fase y su propósito.
    \item Definir cuándo se pasa a la siguiente (triggers).
    \item Escribir instrucciones detalladas para cada evento.
\end{enumerate}

Por ejemplo, el modelo 5E (Engage, Explore, Explain, Elaborate, Evaluate) para indagación científica:

\begin{lstlisting}
model:
  id: "5e-inquiry"
  name: "5E Inquiry Model"
  philosophy: "Discovery through guided 
               exploration"
  
  events:
    - id: "E1_ENGAGE"
      name: "Engage"
      instructions: |
        Captura atencion con pregunta
        provocadora o fenomeno intrigante.
        NO reveles la respuesta todavia.
      transition_trigger:
        condition: "student_response_received"
        next_event: "E2_EXPLORE"
    
    - id: "E2_EXPLORE"
      name: "Explore"
      instructions: |
        Guia al estudiante a descubrir
        por si mismo mediante preguntas
        orientadoras. NO des respuestas
        directas en esta fase.
      transition_trigger:
        condition: "exploration_complete"
        next_event: "E3_EXPLAIN"
\end{lstlisting}

La creación de modelos personalizados abre la posibilidad de un \textbf{repositorio comunitario} donde diseñadores instruccionales compartan metodologías validadas que otros docentes puedan aplicar directamente a sus cursos.

\subsection{Limitaciones de los Modelos}

Es importante reconocer las limitaciones de los modelos instruccionales en \tdl{}:

\begin{itemize}
    \item \textbf{Secuencia predefinida}: Los modelos definen secuencias que se repiten para cada unidad. No hay ramificación condicional basada en el desempeño del estudiante.
    \item \textbf{Sin diagnóstico cognitivo}: Los modelos no detectan misconceptions específicas más allá de lo que el LLM puede inferir conversacionalmente.
    \item \textbf{Dependencia del LLM}: La calidad de ejecución depende de cuán fielmente el LLM siga las instrucciones. Modelos más capaces (GPT-4, Claude 3) producen mejores resultados.
\end{itemize}

Estas limitaciones son deliberadas: mantienen \tdl{} simple y portable. Funcionalidades más avanzadas requerirían infraestructura que contradice el objetivo de despliegue sobre plataformas comerciales existentes.

\input{sections/07-portabilidad}
\input{sections/08-discusion}
% ============================================================================
% IX. CONCLUSIONES Y TRABAJO FUTURO
% ============================================================================

\section{Conclusiones y Trabajo Futuro}

\subsection{Resumen de Contribuciones}

Este artículo ha presentado \tdl{} (Tutor Description Language), una evolución de \adl{} que implementa una arquitectura de cuatro capas para sistemas de tutoría basados en LLM. Las contribuciones principales son:

\begin{enumerate}
    \item \textbf{Arquitectura desacoplada}: Separación de Engine, Modelo Instruccional, Secuencia de Aprendizaje y Contenido Fuente, permitiendo evolución y reutilización independientes.
    
    \item \textbf{Modelo instruccional como componente explícito}: Formalización del ``cómo enseñar'' como secuencia de eventos instruccionales reutilizables, alineados con teorías de Gagné y Bloom.
    
    \item \textbf{Separación metodología-contenido}: El mismo modelo instruccional puede aplicarse a cursos de diferentes disciplinas; el mismo contenido puede enseñarse con diferentes metodologías.
    
    \item \textbf{Evolución documentada desde ADL}: Análisis de las limitaciones de \adl{} para tutorías y principios de diseño que guiaron \tdl{}.
    
    \item \textbf{Especificación formal y herramientas}: Esquemas JSON Schema y validador Python para garantizar correctitud antes del despliegue.
    
    \item \textbf{Portabilidad multiplataforma}: Los mismos archivos \tdl{} funcionan en ChatGPT, Claude, Gemini y OpenWebUI sin modificación.
\end{enumerate}

\tdl{} no pretende ser conceptualmente novedoso, sino pragmáticamente útil: una implementación accesible de principios establecidos, adaptada a la realidad de los LLM actuales.

\subsection{Agenda de Investigación}

La validación empírica de \tdl{} requiere responder preguntas de investigación específicas:

\begin{itemize}
    \item \textbf{RQ1}: ¿Producen los tutores \tdl{} mayor ganancia de aprendizaje que LLM sin estructura pedagógica formal?
    \item \textbf{RQ2}: ¿Reduce \tdl{} la carga docente manteniendo o mejorando la calidad de la atención individualizada?
    \item \textbf{RQ3}: ¿Pueden docentes sin formación técnica avanzada crear tutores \tdl{} funcionales en tiempo razonable?
    \item \textbf{RQ4}: ¿Qué modelos instruccionales son más efectivos para qué tipos de contenido y población estudiantil?
\end{itemize}

\subsection{Hipótesis}

Basándonos en la literatura revisada, proponemos las siguientes hipótesis:

\begin{itemize}
    \item \textbf{H1}: Tutores \tdl{} con modelo Bloom 8-Step producirán mayor ganancia de aprendizaje que LLM con prompt genérico, en contenido técnico.
    
    \item \textbf{H2}: Docentes con tutores \tdl{} desplegados reportarán menor tiempo dedicado a atención individualizada repetitiva.
    
    \item \textbf{H3}: Docentes sin experiencia previa lograrán crear tutores \tdl{} funcionales en menos de 2 horas de trabajo.
    
    \item \textbf{H4}: Los modelos interactivos serán preferidos para contenido conceptual; los expositivos para contenido normativo.
\end{itemize}

\subsection{Diseño Experimental Propuesto}

Planificamos un estudio cuasi-experimental durante el curso 2025-2026 con tres condiciones:

\begin{enumerate}
    \item \textbf{Grupo TDL}: Estudiantes con acceso a tutor \tdl{} con modelo Bloom 8-Step.
    \item \textbf{Grupo LLM genérico}: Estudiantes con acceso a ChatGPT/Claude con prompt básico.
    \item \textbf{Grupo control}: Estudiantes con materiales tradicionales (apuntes, vídeos).
\end{enumerate}

Métricas:

\begin{itemize}
    \item Ganancia normalizada pre-post en tests de conocimiento.
    \item Tiempo de estudio auto-reportado.
    \item Engagement (número y profundidad de interacciones con el tutor).
    \item Satisfacción estudiantil (escala Likert).
    \item Carga docente (horas dedicadas a atención individualizada).
\end{itemize}

\subsection{Trabajo Futuro}

Identificamos varias direcciones de trabajo futuro:

\textbf{Editor visual (TDL Maker)}: Desarrollo de una herramienta web que permita diseñar secuencias de aprendizaje visualmente, generando el \yamlformat{} automáticamente. Esto eliminaría la barrera sintáctica para docentes sin experiencia técnica.

\textbf{Learning analytics}: Integración de hooks para registrar qué eventos se ejecutan, tiempos de respuesta del estudiante, patrones de interacción, y puntos de abandono. Estos datos informarían la mejora iterativa de modelos y secuencias.

\textbf{Repositorio comunitario}: Creación de un repositorio abierto de modelos instruccionales validados, donde diseñadores puedan compartir metodologías y docentes puedan descubrir modelos apropiados para sus necesidades.

\textbf{Extensiones para evaluación}: Añadir secciones para rúbricas de evaluación, criterios de calificación, y generación automática de informes de progreso estudiantil.

\textbf{Student model ligero}: Explorar la viabilidad de incorporar un modelo del estudiante simplificado, aprovechando la memoria conversacional de los LLM o almacenamiento externo, para habilitar cierta adaptación basada en historial.

\textbf{Integración con LMS}: Desarrollar conectores para integrar tutores \tdl{} con sistemas de gestión del aprendizaje (Moodle, Canvas), permitiendo autenticación, seguimiento de progreso, y sincronización de calificaciones.

\subsection{Conclusión Final}

La promesa de escalar la educación personalizada mediante IA no se cumplirá solo con modelos de lenguaje más potentes. Requiere métodos para que los educadores transfieran su expertise pedagógica a estos sistemas. \tdl{} ofrece un camino pragmático hacia este objetivo: permite que el ``cómo enseñar'' se formalice como conocimiento reutilizable mientras el ``qué enseñar'' permanece bajo control del docente.

Al separar responsabilidades en capas bien definidas y alinearse con teorías establecidas de diseño instruccional, \tdl{} facilita la colaboración entre diseñadores instruccionales, docentes y expertos en contenido. El resultado es un marco que amplifica el impacto de los educadores sin comprometer su autoría pedagógica.

La validación empírica determinará si las lecciones del pasado (particularmente de IMS Learning Design) han sido aprendidas. \tdl{} es una propuesta en esa dirección, sujeta a escrutinio experimental y mejora iterativa.


% ============================================================================
% AGRADECIMIENTOS
% ============================================================================

\section*{Agradecimientos}
Este trabajo ha sido parcialmente financiado por el proyecto AI4ProSa.

% ============================================================================
% BIBLIOGRAFÍA
% ============================================================================

\input{sections/referencias}

\end{document}
